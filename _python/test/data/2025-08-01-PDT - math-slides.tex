\documentclass[17pt,landscape]{foils}
\TITLEFOIL{AAA}{BBB}
\newcommand\pdfpsfrag{no}
\newcommand\foiltex{yes}
\newcommand\showincomplete{no}
\def\allslides{yes}
\newcommand{\isp}{\item}
%% users inputting this file should define below yes/no commands

%\def\pdfpsfrag{yes or no}
%\def\reusepsfragpdf{yes or no}
%\def\foiltex{yes or no}
%\def\algfontmode{yes or no}

%% yes/no conditional 

\usepackage{ifthen}
\newcommand\yesnoexec[3]{
\ifthenelse{\equal{#1}{yes}}
{#2%
}{
	\ifthenelse{\equal{#1}{no}}{#3}{\errmessage{{#1} should be either "yes" or "no"}}
}
}

\ifdefined\pdfpsfrag \else
\def\pdfpsfrag{no}
\fi

\yesnoexec{\pdfpsfrag}{%
\ifdefined\reusepsfragpdf \else
\def\reusepsfragpdf{no}
\fi
}{}

\ifdefined\foiltex \else
\def\foiltex{no}
\fi

\ifdefined\algfontmode \else
\def\algfontmode{no}
\fi

\yesnoexec{\algfontmode}{%
%\usepackage[subscriptcorrection,slantedGreek,nofontinfo,mtpcal,mtphrb,mtpccal,mtpscr,mtpfrak]{mtpro2}
\usepackage[mtpccal,mtpscr]{mtpro2}
%\usepackage[mtpccal,mtpfrak,mtpscr]{mtpro2}
%\usepackage[mathscr]{euscript} % mathscr
\usepackage{amsfonts} % mathfrak

\newcommand\mathalgfont[1]{\mathcal{#1}}
\newcommand\mathcalfont[1]{\mathscr{#1}}
}{%
\usepackage{amssymb} % mathfrak
\usepackage{mathrsfs} % mathscr

\newcommand\mathalgfont[1]{\mathscr{#1}}
\newcommand\mathcalfont[1]{\mathcal{#1}}
}


%% PICTURE ENVIRONMENT

\newcommand\rect[6]{%
\put(#1,#2){\line(1,0){#5}}%
\put(#1,#2){\line(0,1){#6}}%
\put(#1,#4){\line(1,0){#5}}%
\put(#3,#2){\line(0,1){#6}}%
}

\newcommand\crect[7]{%
\put(#1,#2){\color{#7}\line(1,0){#5}}%
\put(#1,#2){\color{#7}\line(0,1){#6}}%
\put(#1,#4){\color{#7}\line(1,0){#5}}%
\put(#3,#2){\color{#7}\line(0,1){#6}}%
}

\newcommand\pptext[3]{
	\put(#1){\makebox(0,0)[#2]{#3}}
}

\newcommand\ptext[4]{\pptext{#1}{#2}{\textcolor{#4}{#3}}}


%% COLOR DEFINITIONS

%\usepackage[]{hyperref} % i put below due to conflict between hyperref package and \newtheorem\theorem
\usepackage[
  colorlinks=true,
  linkcolor=black,
  urlcolor=blue,
  citecolor=blue,
%  pdftitle={Your Document Title},
  pdfauthor={Sunghee Yun}
]{hyperref}

\usepackage[dvipsnames]{xcolor} % for below color definitions

\definecolor{rgb-60-60-60}{rgb}{.6, .6, .6}
\definecolor{rgb-10-80-10}{rgb}{.1, .8, .1}
\definecolor{crimson}{rgb}{0.86, 0.08, 0.24}
\definecolor{dimgray}{rgb}{0.41, 0.41, 0.41}
\definecolor{lavendergray}{rgb}{0.77, 0.76, 0.82}
\definecolor{cobalt}{rgb}{0.0, 0.28, 0.67}
\definecolor{unitednationsblue}{rgb}{0.36, 0.57, 0.9}
\definecolor{ultramarine}{rgb}{0.07, 0.04, 0.56}
\definecolor{ultramarineblue}{rgb}{0.25, 0.4, 0.96}
\definecolor{ufogreen}{rgb}{0.24, 0.82, 0.44}
\definecolor{tropicalrainforest}{rgb}{0.0, 0.46, 0.37}
\definecolor{tiffanyblue}{rgb}{0.04, 0.73, 0.71}
\definecolor{springgreen}{rgb}{0.0, 1.0, 0.5}
\definecolor{Lime-Green}{HTML}{32cd32}
\definecolor{saddle-brown}{RGB}{139,69,19}
\definecolor{sienna}{RGB}{160,82,45}
\definecolor{chocolate}{RGB}{210,105,30}

% SOME COMMANDS & ENVIRONMENTS FOR MAKING LECTURE NOTES

\newcounter{oursection}
\newcommand{\oursection}[1]{
 \addtocounter{oursection}{1}
 \setcounter{equation}{0}
 \clearpage \begin{center} {\Huge\bfseries #1} \end{center}
 {\vspace*{0.15cm} \hrule height.3mm} \bigskip
 \addcontentsline{toc}{section}{#1}
}
\newcommand{\oursectionf}[1]{  % for use with foiltex
 \addtocounter{oursection}{1}
 \setcounter{equation}{0}
 \foilhead[-.5cm]{#1 \vspace*{0.8cm} \hrule height.3mm }
 \LogoOn
}
\newcommand{\oursectionfl}[1]{  % for use with foiltex landscape
 \addtocounter{oursection}{1}
 \setcounter{equation}{0}
 \foilhead[-1.0cm]{#1}
 \LogoOn
}

\newcounter{lecture}
\newcommand{\lecture}[1]{
 \addtocounter{lecture}{1}
 \setcounter{equation}{0}
 \setcounter{page}{1}
 \renewcommand{\theequation}{\arabic{equation}}
 \renewcommand{\thepage}{\arabic{lecture} -- \arabic{page}}
 \raggedright \sffamily \LARGE
 \cleardoublepage\begin{center}
 {\Huge\bfseries Lecture \arabic{lecture} \bigskip \\ #1}\end{center}
 {\vspace*{0.15cm} \hrule height.3mm} \bigskip
 \addcontentsline{toc}{chapter}{\protect\numberline{\arabic{lecture}}{#1}}
 \pagestyle{myheadings}
 \markboth{Lecture \arabic{lecture}}{#1}
}
\newcommand{\lecturef}[1]{
 \addtocounter{lecture}{1}
 \setcounter{equation}{0}
 \setcounter{page}{1}
 \renewcommand{\theequation}{\arabic{equation}}
 \renewcommand{\thepage}{\arabic{lecture}--\arabic{page}}
 \parindent 0pt
 \MyLogo{#1}
 \rightfooter{\thepage}
 \leftheader{}
 \rightheader{}
 \LogoOff
 \begin{center}
 {\large\bfseries Lecture \arabic{lecture} \bigskip \\ #1}
 \end{center}
 {\vspace*{0.8cm} \hrule height.3mm}
 \bigskip
}
\newcommand{\lecturefl}[1]{   % use with foiltex landscape
 \addtocounter{lecture}{1}
 \setcounter{equation}{0}
 \setcounter{page}{1}
 \renewcommand{\theequation}{\arabic{equation}}
 \renewcommand{\thepage}{\arabic{lecture}--\arabic{page}}
 \addtolength{\topmargin}{-1.5cm}
 \raggedright
 \parindent 0pt
 \rightfooter{\thepage}
 \leftheader{}
 \rightheader{}
 \LogoOff
 \begin{center}
 {\Large \bfseries Lecture \arabic{lecture} \\*[\bigskipamount] {#1}}
 \end{center}
 \MyLogo{#1}
}


% NORMAL SENTENCES

% acronyms

\newcommand{\eg}{{\it e.g.}}
\newcommand{\ie}{{\it i.e.}}
\newcommand{\etc}{{\it etc.}}
\newcommand{\cf}{{\it cf.}}

\newcommand{\graystrikethrough}[1]{\textcolor{rgb-60-60-60}{\sout{#1}}}

\newcommand\blfootnote[1]{%
	\begingroup
	\renewcommand\thefootnote{}\footnote{#1}%
	\addtocounter{footnote}{-1}%
	\endgroup
}

\newcommand\brfootnote[1]{\let\thefootnote\relax\footnotetext}

% THEOREMS AND SUCH

\usepackage{thmtools}

\yesnoexec{\foiltex}{%
	\declaretheorem{axiom}
	\declaretheorem{law}
	\declaretheorem{principle}
	\declaretheorem{definition}
	\declaretheorem{theorem}
	\declaretheorem{lemma}
	\declaretheorem{proposition}
	\declaretheorem{corollary}
	\declaretheorem{conjecture}
	\declaretheorem{inequality}
	\declaretheorem{formula}
	\declaretheorem{algorithm}
}{%
	\declaretheorem[numberwithin=section]{axiom}
	\declaretheorem[numberwithin=section]{law}
	\declaretheorem[numberwithin=section]{principle}
	\declaretheorem[numberwithin=section]{definition}
	\declaretheorem[numberwithin=section]{theorem}
	\declaretheorem[numberwithin=section]{lemma}
	\declaretheorem[numberwithin=section]{proposition}
	\declaretheorem[numberwithin=section]{corollary}
	\declaretheorem[numberwithin=section]{conjecture}
	\declaretheorem[numberwithin=section]{inequality}
	\declaretheorem[numberwithin=section]{formula}
	\declaretheorem[numberwithin=section]{algorithm}
}

\newcommand\axiomname{Axiom}
\newcommand\lawname{Law}
\newcommand\principlename{Principle}
\newcommand\definitionname{Definition}
\newcommand\theoremname{Theorem}
\newcommand\lemmaname{Lemma}
\newcommand\propositionname{Proposition}
\newcommand\corollaryname{Corollary}
\newcommand\conjecturename{Conjecture}
\newcommand\inequalityname{Ineq}
\newcommand\formulaname{Formula}
\newcommand\algorithmname{Algorithm}

% alias
\newcommand\definename{\definitionname}

\newenvironment{proof}{\begin{quote}\textit{Proof}:}{\end{quote}}
\newenvironment{solution}{\begin{quote}\textit{Solution}:}{\end{quote}}
\newenvironment{pcode}{\begin{quote}\textit{Python source code}:}{\end{quote}}
\newenvironment{names}{\begin{itshape}}{\end{itshape}}

\newcommand{\qed}{{\bf Q.E.D.}}
\renewcommand{\qed}{\rule[-.5ex]{.5em}{2ex}}
\newcommand{\textfn}[1]{\textsl{#1}}

% table

\newcommand{\tparbox}[2]{%
{\parbox[c]{#1}{\center\vspace{-.4\baselineskip}{#2}\vspace{.3\baselineskip}}}}

% software

\newenvironment{code}{\begin{quote}\begin{tt}}{\end{tt}\end{quote}}

% math

\newcommand{\pluseq}{\mathrel{+}=}

\newcommand{\bmyeq}{\[}
\newcommand{\emyeq}{\]}
\newcommand{\bmyeql}[1]{\begin{equation}\label{#1}}
\newcommand{\emyeql}{\end{equation}}
%\newenvironment{myeq}{\[}{\]}
%\newenvironment{myeql}[1]{\begin{equation}\label{#1}}{\end{equation}}

\newcommand{\onehalf}{\ensuremath{\frac{1}{2}}}
\newcommand{\onethird}{\ensuremath{\frac{1}{3}}}
\newcommand{\onefourth}{\frac{1}{4}}
\newcommand{\sumft}[2]{\sum_{#1}^{#2}}
\newcommand{\sumioneton}{\sumionetok{n}}
\newcommand{\sumionetok}[1]{\sum_{i=1}^#1}
\newcommand{\sumoneto}[2]{\sum_{#1=1}^{#2}}
\newcommand{\sumoneton}[1]{\sumoneto{#1}{n}}
\newcommand{\prodoneto}[2]{\prod_{#1=1}^{#2}}
\newcommand{\prodoneton}[1]{\prodoneto{#1}{n}}

\newcommand{\listoneto}[1]{\ensuremath{1,2,\ldots,#1}}
\newcommand{\diagxoneto}[2]{\ensuremath{\diag({#1}_1,{#1}_2,\ldots,{#1}_{#2})}}
\newcommand{\setxoneto}[2]{\ensuremath{\{\listxoneto{#1}{#2}\}}}
\newcommand{\listxoneto}[2]{\ensuremath{{#1}_1,{#1}_2,\ldots,{#1}_{#2}}}
\newcommand{\setoneto}[1]{\ensuremath{\{1,2,\ldots,#1\}}}

\newcommand{\diagmat}[2]{\diagoneto{#1}{#2}}

\newcommand{\setoneton}[1]{\setoneton{#1}}
\newcommand{\setxtok}[2]{\setxoneto{#1}{#2}}

% matrices

\newcommand{\colvectwo}[2]{\ensuremath{\begin{my-matrix}{c}{#1}\\{#2}\end{my-matrix}}}
\newcommand{\colvecthree}[3]{\ensuremath{\begin{my-matrix}{c}{#1}\\{#2}\\{#3}\end{my-matrix}}}
\newcommand{\colvecfour}[4]{\ensuremath{\begin{my-matrix}{c}{#1}\\{#2}\\{#3}\\{#4}\end{my-matrix}}}
\newcommand{\rowvectwo}[2]{\ensuremath{\begin{my-matrix}{cc}{#1}&{#2}\end{my-matrix}}}
\newcommand{\rowvecthree}[3]{\ensuremath{\begin{my-matrix}{ccc}{#1}&{#2}&{#3}\end{my-matrix}}}
\newcommand{\rowvecfour}[4]{\ensuremath{\begin{my-matrix}{cccc}{#1}&{#2}&{#3}&{#4}\end{my-matrix}}}
\newcommand{\diagtwo}[2]{\ensuremath{\begin{my-matrix}{cc}{#1}&0\\0& {#2}\end{my-matrix}}}
\newcommand{\mattwotwo}[4]{\ensuremath{\begin{my-matrix}{cc}{#1}&{#2}\\{#3}&{#4}\end{my-matrix}}}
\newcommand{\bigmat}[9]{\ensuremath{\begin{my-matrix}{cccc} #1&#2&\cdots&#3\\ #4&#5&\cdots&#6\\ \vdots&\vdots&\ddots&\vdots\\ #7&#8&\cdots&#9 \end{my-matrix}}}
%\newcommand{\matdotff}[9]{\matff{#1}{#2}{\cdots}{#3}{#4}{#5}{\cdots}{#6}{\vdots}{\vdots}{\ddots}{\vdots}{#7}{#8}{\cdots}{#9}}

\newcommand{\matthreethree}[9]{%
	\begin{my-matrix}{ccc}%
	{#1}&{#2}&{#3}%
	\\{#4}&{#5}&{#6}%
	\\{#7}&{#8}&{#9}%
	\end{my-matrix}%
}
\newcommand{\matthreethreeT}[9]{%
	\matthreethree%
	{#1}{#4}{#7}%
	{#2}{#5}{#8}%
	{#3}{#6}{#9}%
}

\newenvironment{my-matrix}{\left[\begin{array}}{\end{array}\right]}

\newcommand{\mbyn}[2]{\ensuremath{#1\times #2}}
\newcommand{\realmat}[2]{\ensuremath{\reals^{\mbyn{#1}{#2}}}}
\newcommand{\realsqmat}[1]{\ensuremath{\reals^{\mbyn{#1}{#1}}}}
\newcommand{\defequal}{\triangleq}
\newcommand\symset[1]{\ensuremath{\mbox{\bf S}^{#1}}}
\newcommand\possemidefset[1]{\ensuremath{\mbox{\bf S}_+^{#1}}}
\newcommand\posdefset[1]{\ensuremath{\mbox{\bf S}_{++}^{#1}}}

% spaces for integral, etc.

\newcommand{\dspace}{\,}
\newcommand{\dx}{{\dspace dx}}
\newcommand{\dy}{{\dspace dy}}
\newcommand{\dt}{{\dspace dt}}
\newcommand{\intspace}{\!\!}
\newcommand{\sqrtspace}{\,}
\newcommand{\aftersqrtspace}{\sqrtspace}
\newcommand{\dividespace}{\!}


\usepackage{amsmath}

%\DeclareMathOperator\support{support}
\newcommand\support{\operatorname*{\bf support}}
%\newcommand\support{{\mathop{\bf support}}}


\DeclareMathOperator\Img{Im}
\DeclareMathOperator\Ker{Ker}
\DeclareMathOperator\Gal{Gal}
\DeclareMathOperator\Map{Map}
\DeclareMathOperator\Aut{Aut}
\DeclareMathOperator\End{End}
\DeclareMathOperator\Irr{Irr}
\DeclareMathOperator\ev{ev}
\DeclareMathOperator\affinehull{\bf aff}
\DeclareMathOperator\relint{\bf relint}
\DeclareMathOperator\cvxhull{\bf Conv}
\DeclareMathOperator\boundary{\bf bd}
\DeclareMathOperator\epi{\bf epi}
\DeclareMathOperator\hypo{\bf hypo}

\newcommand{\arginf}{\mathop{\mathrm{arginf}}}
\newcommand{\argsup}{\mathop{\mathrm {argsup}}}
\newcommand{\argmin}{\mathop{\mathrm {argmin}}}
\newcommand{\argmax}{\mathop{\mathrm {argmax}}}

\newcommand\ereals{\reals\cup\{-\infty,\infty\}}
\newcommand{\reals}{\ensuremath{\mbox{\bf R}}}
\newcommand{\preals}{\ensuremath{\reals_{+}}}
\newcommand{\prealk}[1]{\ensuremath{\reals_{+}^{#1}}}
\newcommand{\ppreals}{\ensuremath{\reals_{++}}}
\newcommand{\pprealk}[1]{\ensuremath{\reals_{++}^{#1}}}
\newcommand{\complexes}{\ensuremath{\mbox{\bf C}}}
\newcommand{\integers}{{\mbox{\bf Z}}}
\newcommand{\naturals}{{\mbox{\bf N}}}
\newcommand{\rationals}{{\ensuremath{\mbox{\bf Q}}}}

\newcommand{\realspace}[2]{\reals^{#1\times #2}}
\newcommand{\compspace}[2]{\complexes^{#1\times #2}}

\newcommand{\identity}{\mbox{\bf I}}
\newcommand{\nullspace}{{\mathcalfont N}}
\newcommand{\range}{{\mathcalfont R}}

% vectors, sets, etc.

\newcommand{\one}{\mathbf{1}}
\newcommand{\ones}{\mathbf 1}
\newcommand{\ordinal}{^{\mathrm{th}}}
\newcommand{\set}[2]{\ensuremath{\{#1|#2\}}}
\newcommand{\bigset}[2]{\ensuremath{\left\{{#1}\left|{#2}\right.\right\}}}
\newcommand{\bigsetl}[2]{\left\{\left.{#1}\right|{#2}\right\}}

% operators

\newcommand{\Expect}{\mathop{\bf E{}}}
\newcommand{\Var}{\mathop{\bf  Var{}}}
\newcommand{\Cov}{\mathop{\bf Cov}}
\newcommand{\Prob}{\mathop{\bf Prob}}
%\newcommand\Prob{\operatorname*{\bf P}}
\newcommand\prob[1]{\Prob\left\{#1\right\}}
\renewcommand\prob[1]{\Prob\left(#1\right)}

\newcommand{\smallo}{{\mathop{\bf o}}}

\newcommand{\jac}{{\mathcalfont{J}}}
\newcommand{\diag}{\mathop{\bf diag}}
\newcommand{\Rank}{\mathop{\bf Rank}}
\newcommand{\rank}{\mathop{\bf rank}}
\newcommand{\dimn}{\mathop{\bf dim}}
\newcommand{\Tr}{\mathop{\bf Tr}} % trace
\newcommand{\dom}{\mathop{\bf dom}}
\newcommand{\Det}{\det}
\newcommand{\adj}{\mathop{\bf adj}}
\newcommand{\minor}{\mathop{\bf minor}}
%\newcommand{\Det}{{\mathop{\bf Det}}}
%\newcommand{\determinant}[1]{|#1|}
\newcommand{\sign}{{\mathop{\bf sign}}}
\newcommand{\dist}{{\mathop{\bf dist}}}

% probability space

\newcommand{\probsubset}{{\mathcalfont{P}}}
\newcommand{\eset}{{\mathcalfont{E}}}

\newcommand{\probspace}{{\Omega}}

% optimization

% phrases

\newcommand\iaoi{\emph{if and only if}}
\newcommand\wrt{with respect to}

% mathematicians' names

\newcommand\cara{Carath\'{e}odory}

% FOR ANALYSIS
% names of families, collections, sets, etc.

\newcommand{\group}[2]{\ensuremath{(#1,#2)}}
\newcommand{\generates}[1]{\ensuremath{\langle {#1} \rangle}}
\newcommand{\generatest}[1]{\ensuremath{\left\langle {#1} \right\rangle}}
\newcommand{\perm}[1]{\ensuremath{\mathrm{Perm}(#1)}} % permutations
\newcommand{\aut}[1]{\ensuremath{\mathrm{Aut}(#1)}} % set of automorphisms (as a group)
\newcommand{\injhomeo}{\hookrightarrow}
\newcommand{\isomorph}{\approx}
\newcommand{\ideal}[1]{\ensuremath{\mathfrak{#1}}}

\newcommand{\collk}[1]{\ensuremath{{\mathcalfont{#1}}}} % collection
\newcommand{\classk}[1]{\ensuremath{\collk{#1}}} % class
\newcommand{\algk}[1]{\ensuremath{\mathalgfont{#1}}} % algebra
\newcommand{\metrics}[2]{\ensuremath{\langle {#1}, {#2}\rangle}} % metric space

\newcommand{\topol}[1]{\ensuremath{\mathfrak{#1}}} % topology
\newcommand{\topos}[2]{\ensuremath{{\langle {#1}, \topol{#2}\rangle}}} % topological space

\newcommand{\measu}[2]{\ensuremath{({#1}, {#2})}} % measurable space
\newcommand{\meas}[3]{\ensuremath{({#1}, {#2}, {#3})}} % measure space
\newcommand{\meast}[3]{\ensuremath{\left({#1}, {#2}, {#3}\right)}} % measure space

\newcommand{\powerset}{\mathcalfont{P}}
\newcommand{\field}{\mbox{\bf F}}
\newcommand\primefield[1]{\ensuremath{\field_{#1}}}
\newcommand\finitefield[2]{\ensuremath{\field_{{#1}^{#2}}}}
\newcommand\frobmap[2]{\ensuremath{\varphi_{{#1},{#2}}}}
\newcommand{\compl}[1]{\ensuremath{\tilde{#1}}} % set complement
\newcommand{\interior}[1]{\ensuremath{{#1}^\circ}} % set interior
\newcommand{\subsetset}[1]{\ensuremath{\mathcalfont{#1}}} % set of subsets

\newcommand{\pair}[2]{\ensuremath{{\langle {#1}, {#2}\rangle}}}
\newcommand{\innerp}[2]{\ensuremath{\langle{#1},{#2}\rangle}} % inner product
\newcommand{\innerpt}[2]{\ensuremath{\left\langle{#1},{#2}\right\rangle}} % inner product - tall version
\newcommand{\dimext}[2]{\ensuremath{[{#1}:{#2}]}}

\newcommand\restrict[2]{\ensuremath{{#1}|{#2}}}
\newcommand\algclosure[1]{\ensuremath{{#1}^\mathrm{a}}}
\newcommand\sepclosure[1]{\ensuremath{{#1}^\mathrm{sep}}}
\newcommand\maxabext[1]{\ensuremath{{#1}^\mathrm{ab}}}

\newcommand\ball[2]{\ensuremath{B(#1,#2)}}

% optimization

\newcommand\optfdk[2]{\ensuremath{{#1}^\mathrm{#2}}}
\newcommand\tildeoptfdk[2]{\ensuremath{{\tilde{#1}}^\mathrm{#2}}}
\newcommand\fobj{\optfdk{f}{obj}}
\newcommand\fie{\optfdk{f}{ie}}
\newcommand\feq{\optfdk{f}{eq}}
\newcommand\tildefobj{\tildeoptfdk{f}{obj}}
\newcommand\tildefie{\tildeoptfdk{f}{ie}}
\newcommand\tildefeq{\tildeoptfdk{f}{eq}}

\newcommand\xdomain{\ensuremath{\mathcalfont{X}}}
\newcommand\xobj{\optfdk{\xdomain}{obj}}
\newcommand\xie{\optfdk{\xdomain}{ie}}
\newcommand\xeq{\optfdk{\xdomain}{eq}}

\newcommand\optdomain{\ensuremath{\mathcalfont{D}}}
\newcommand\optfeasset{\ensuremath{\mathcalfont{F}}}

\def\DeltaSirDir{yes}
\newcommand\sdirletter[2]{\ifthenelse{\equal{\DeltaSirDir}{yes}}{\ensuremath{\Delta #1}}{\ensuremath{#2}}}
\newcommand\seqk[2]{\ensuremath{{#1}^{(#2)}}}
\newcommand\xseqk[1]{\seqk{x}{#1}}
\newcommand\nuseqk[1]{\seqk{\nu}{#1}}
\newcommand\lbdseqk[1]{\seqk{\lambda}{#1}}
\newcommand\sdir{\sdirletter{x}{v}}
\newcommand\sdirlbd{\sdirletter{\lambda}{\Delta \lambda}}
\newcommand\sdirnu{\sdirletter{\nu}{w}}
\newcommand\sdiry{\sdirletter{y}{\Delta y}}
\newcommand\ntsdir{\ensuremath{\sdir_\mathrm{nt}}}
\newcommand\pdsdir{\ensuremath{\sdir_\mathrm{pd}}}
\newcommand\ntsdirnu{\ensuremath{\sdirnu_\mathrm{nt}}}
\newcommand\pdsdirnu{\ensuremath{\sdirnu_\mathrm{pd}}}
\newcommand\pdsdirlbd{\ensuremath{\sdirlbd_\mathrm{pd}}}
\newcommand\pdsdiry{\ensuremath{\sdiry_\mathrm{pd}}}
\newcommand\sdirk[1]{\seqk{\sdir}{#1}}
\newcommand\slen{\ensuremath{t}}
\newcommand\slenk[1]{\seqk{\slen}{#1}}

\newcommand\onelineoptprob[3]{\ensuremath{%
	\mbox{#1}\;\;{#2}
%	\;\;
	\ifthenelse{\equal{#3}{}}{}{%
	\quad%
	\mbox{s.t.}\;%
	{#3}%
}}}

% mo, relation, sequence, indexed collection, etc.

\newcommand{\Mod}[1]{\ (\mathrm{mod}\ #1)}
\newcommand{\rel}{\mathbf{R}}
\newcommand{\relxy}[2]{{#1}\ \rel\ {#2}}
\newcommand{\seq}[1]{\ensuremath{{\left\langle{#1}\right\rangle}}}
\newcommand{\seqscr}[3]{\ensuremath{\seq{#1}_{#2}^{#3}}}
\newcommand{\indexedcol}[1]{\ensuremath{{\{{#1}\}}}}

% CLOSURE

%command for alg-closure that automatically adapts its 'bar' to the arg based on the args length (including '\')
\newcommand{\ols}[1]{\mskip.5\thinmuskip\overline{\mskip-.5\thinmuskip {#1} \mskip-.5\thinmuskip}\mskip.5\thinmuskip} % overline short
\newcommand{\olsi}[1]{\,\overline{\!{#1}}} % overline short italic
\makeatletter
\newcommand\closure[1]{\ensuremath{%
	\tctestifnum{\count@stringtoks{#1}>1} %checks if number of chars in arg > 1 (including '\')
	{\ols{#1}} %if arg is longer than just one char, e.g. \mathbb{Q}, \mathbb{F},...
	{\olsi{#1}} %if arg is just one char, e.g. K, L,...
}%
}
% FROM TOKCYCLE:
\long\def\count@stringtoks#1{\tc@earg\count@toks{\string#1}}
\long\def\count@toks#1{\the\numexpr-1\count@@toks#1.\tc@endcnt}
\long\def\count@@toks#1#2\tc@endcnt{+1\tc@ifempty{#2}{\relax}{\count@@toks#2\tc@endcnt}}
\def\tc@ifempty#1{\tc@testxifx{\expandafter\relax\detokenize{#1}\relax}}
\long\def\tc@earg#1#2{\expandafter#1\expandafter{#2}}
\long\def\tctestifnum#1{\tctestifcon{\ifnum#1\relax}}
\long\def\tctestifcon#1{#1\expandafter\tc@exfirst\else\expandafter\tc@exsecond\fi}
\long\def\tc@testxifx{\tc@earg\tctestifx}
\long\def\tctestifx#1{\tctestifcon{\ifx#1}}
\long\def\tc@exfirst#1#2{#1}
\long\def\tc@exsecond#1#2{#2}
\makeatother
%


% graphics - include figures

%\newcommand\includefig[2]{\yesnoexec{\pdfpsfrag}{\includegraphics[#2]{#1}}{\includegraphics[#2]{#1}}}
\newcommand\mypsfrag[2]{\yesnoexec{\pdfpsfrag}{\psfrag{#1}{#2}}{}}

\newcommand\puttwofigs[2]{%
	\begin{center}
	\hfill
	{#1}
	\hfill
	{#2}
	\hfill
	\hfill
	\ %
	\end{center}
}

\newcommand\puttworoundedfigs[3]{%
	\begin{center}
	\hfill
	{\roundedbox{#1}{#2}}
	\hfill
	{\roundedbox{#1}{#3}}
	\hfill
	\hfill
	\ %
	\end{center}
}

\newcommand\roundedbox[2]{%
\begin{tikzpicture}%
\sbox0{#2}%
\path[clip,draw,rounded corners=#1] (0,0) rectangle (\wd0,\ht0);%
\path (0.5\wd0,0.5\ht0) node[inner sep=0pt]{\usebox0};%
\end{tikzpicture}%
}

% COMMANDS ORIGINALLY INTENDED FOR FOILTEX

\newcommand\specialidxprefix{ZZ}
\newcommand\idximportant[1]{\index{\specialidxprefix-important!#1}}
\newcommand\idxrevisit[1]{\index{\specialidxprefix-revisit!#1}}
\newcommand\idxtodo[1]{\index{\specialidxprefix-todo!#1}}
\newcommand\idxfig[1]{\index{\specialidxprefix-figures!#1}}

% page label and reference
\newcommand\pagelabel{\phantomsection\label}

% enumeration - bullet points

\usepackage{enumitem}
\newcommand\bit{\begin{itemize}}
\newcommand\eit{\end{itemize}}
\newcommand\ibit{\begin{itemize}[leftmargin=1.5em]}
\newcommand\iitem{\item [-]}

% *my* theorems and such

\newcommand\theoremslikepostvspace{\vspace{-.5em}}
\newcommand\theoremslikepostvspacet{\vspace{-1em}}

\newcommand\notempty[1]{\ifthenelse{\not \equal{#1}{}}}

\newcommand\mybegin[3]{%
\notempty{#3}{%
\begin{#1}[#3]%
\index{#3}\index{#2!#3}%
\label{#1:#3}\pagelabel{page:#1:#3}%
}{%
\begin{#1}
}%
}

\newcommand\myend[1]{%
\end{#1}
\theoremslikepostvspace
}

\newenvironment{myaxiom}[1]{\mybegin{axiom}{axioms}{#1}}{\myend{axiom}}
\newenvironment{mylaw}[1]{\mybegin{law}{laws}{#1}}{\myend{law}}
\newenvironment{myprinciple}[1]{\mybegin{principle}{principles}{#1}}{\myend{principle}}
\newenvironment{mydefinition}[1]{\mybegin{definition}{definitions}{#1}}{\myend{definition}}
\newenvironment{mytheorem}[1]{\mybegin{theorem}{theorems}{#1}}{\myend{theorem}}
\newenvironment{mylemma}[1]{\mybegin{lemma}{lemmas}{#1}}{\myend{lemma}}
\newenvironment{myproposition}[1]{\mybegin{proposition}{propositions}{#1}}{\myend{proposition}}
\newenvironment{mycorollary}[1]{\mybegin{corollary}{corollaries}{#1}}{\myend{corollary}}
\newenvironment{myconjecture}[1]{\mybegin{conjecture}{conjectures}{#1}}{\myend{conjecture}}
\newenvironment{myinequality}[1]{\mybegin{inequality}{inequalities}{#1}}{\myend{inequality}}
\newenvironment{myformula}[1]{\mybegin{formula}{formula}{#1}}{\myend{formula}}
\newenvironment{myalgorithm}[1]{\mybegin{algorithm}{algorithms}{#1}}{\myend{algorithm}}

% proof env

\newtheorem{proofenv}{Proof}
\newenvironment{myproof}[1]{%
\item%
\begin{proofenv}%
\label{proof:#1}
\label{proof!#1}
(Proof for ``{#1}'' on page~\pageref{page:statement:#1})%
\end{proofenv}%
}{%
\vspace{.5em}
}

\newcommand\proofref[1]{%
%\textcolor{gray}%
{%
%(Refer to \hyperref[proof:#1]{Proof}~\ref{proof:#1}%
%(proved in \hyperref[proof:#1]{Proof}~\ref{proof:#1}%
(proof can be found in \hyperref[proof:#1]{Proof}~\ref{proof:#1}%
\pagelabel{page:statement:#1})%
}%
}

% terms for definitions, facts, special emphasis, (lemma, theorem, etc.) names

\newcommand\define[1]{\emph{\textcolor{blue}{#1}}}
\newcommand\fact[1]{\emph{\textcolor{Lime-Green}{#1}}}
\newcommand\cemph[1]{\emph{\textcolor{blue}{{#1}}}}
\newcommand\eemph[1]{\emph{\textcolor{crimson}{#1}}}
\newcommand\name[1]{\emph{\textcolor{chocolate}{#1}}}

% equations

\newenvironment{eqn}{%
%
\vspace{-.7em}
\[\\
%
}
{
\]\\
\vspace{-3.4em}\\
}

% when equation is too tall
\newenvironment{teqn}{%
%
\vspace{-.7em}
\[\\
%
}
{
\]\\
\vspace{-3.0em}\\
}

% when equation is too long
\newenvironment{leqn}{%
%
\vspace{-2.0em}
\[\\
%
}
{
\]\\
\vspace{-4.7em}\\
}

\newenvironment{eqna}{%
%
\vspace{-3.7em}
\begin{eqnarray*}\\
%
}
{
\end{eqnarray*}\\
\vspace{-4.5em}\\
}


%% FOR FOILTEX

%%%%%%%%%%%%%%%%%%%%%%%%%%%%%%%%%%%%%%%%%%%%%%%%%%%%%%%%%%%%%%%%%%%%%%%%%%%%%%%%
\yesnoexec{\foiltex}{
\usepackage{mathtools}


%% figure reference

\newcommand\figref[1]{the figure}
\newcommand\foilref[1]{on page~\pageref{foil:#1}}

%% enumeration - bullet points

\renewcommand\bit{\begin{itemize}[leftmargin=1.1em]}
\renewcommand\ibit{\begin{itemize}[leftmargin=2em]}

\newcommand\shrinkspacewithintheoremslikehalf{\vspace{-.5em}}
\newcommand\shrinkspacewithintheoremslike{\vspace{-1em}}
\newcommand\shrinkspacewithintheoremsliket{\vspace{-2em}}

\newcommand\vitem{\vfill \item}
\newcommand\vvitem{\vfill \vfill \item}
\newcommand\viitem{\vfill \iitem}

\newcommand\vfillt {\vfill\vfill}
\newcommand\vfillth{\vfill\vfill\vfill}
\newcommand\vfillf {\vfill\vfill\vfill\vfill}
\newcommand\vfillfi{\vfill\vfill\vfill\vfill\vfill}
\newcommand\vfills {\vfill\vfill\vfill\vfill\vfill\vfill}
\newcommand\vfillse{\vfill\vfill\vfill\vfill\vfill\vfill\vfill}
\newcommand\vfille {\vfill\vfill\vfill\vfill\vfill\vfill\vfill\vfill}
\newcommand\vfilln {\vfill\vfill\vfill\vfill\vfill\vfill\vfill\vfill\vfill}
\newcommand\vfillte{\vfill\vfill\vfill\vfill\vfill\vfill\vfill\vfill\vfill\vfill}


%% index
\usepackage{makeidx}
\makeindex

\setlength{\columnsep}{2em}
\newenvironment{theindex}{%
	\let\item\par
	\newcommand\idxitem{\par\hangindent 40pt}
	\newcommand\subitem{\vspace{-12pt} \idxitem \hspace*{20pt}}
	\newcommand\subsubitem{\vspace{-12pt} \idxitem \hspace*{30pt}}
	\newcommand\indexspace{\par \vskip 3pt \relax}
	\twocolumn
}{%
}

%% slide commands and formatting

\setlength{\parindent}{0in}
\newlength{\lift}
\setlength{\lift}{.5in}
\newlength{\pageheight}
%\setlength{\pageheight}{4in + \lift}
\setlength{\pageheight}{4.5in}

\newcommand{\raiselength}{0in}
\newcommand{\figraiselength}{0in}

\newcommand\labelfoilhead[1]{%
\myfoilhead{#1}%
\pagelabel{foil:#1}%
%\label{foil:#1}%
}

\newcommand\myfoilhead[1]{%
\foilhead{#1}%
\vspace{-\lift}%
}

\newcommand{\titlefoil}[2]{%
\myfoilhead{}%
%\index{#1}
\pagelabel{title-page:#2}
\thispagestyle{empty}%
%\addtocounter{page}{-1}%
%
\vfill%
\begin{center}%
\Large%
\bf%
{#1}
\end{center}%
\vfill%
\MyLogo{\LOGO\ - {\hyperref[title-page:#2]{#1}}}
}

\newcommand\LOGO{\talktitle}

\newcommand{\TITLEFOIL}[2]{%
\myfoilhead{}%
\pagelabel{super-title-page:#2}
%\index{#1}
\thispagestyle{empty}%
%\addtocounter{page}{-1}%
%
\vfill%
\begin{center}%
\huge%
\bf%
{#1}
\end{center}%
\vfill%
\renewcommand\LOGO{\talktitle\ - {\hyperref[super-title-page:#2]{#1}}}
\MyLogo{\LOGO}
}

\newcommand{\nntwocols}[6]{%
\begin{minipage}[t][\pageheight]{#1}\vspace*{0ex}{#6}\end{minipage}%
\hfill%
\begin{minipage}[t]{#2}%
\hfill
\raisebox{-\height+0.7\baselineskip - #4}%
{\hspace*{-3in}{#3}}%
#5
\end{minipage}%
}
\newcommand{\ntwocols}[7]{\nntwocols{#1}{#2}{\includegraphics[#4]{#3}}{#5}{#6}{#7}}

\newcommand{\nntwocolss}[6]{%
\begin{minipage}[t]{#2}%
\hfill
\raisebox{-\height+0.7\baselineskip - #4}%
{#5 \hspace*{-3in}{#3}\hfill\ %
}\end{minipage}%
\hfill%
\begin{minipage}[t][\pageheight]{#1}\vspace*{0ex}{#6}\end{minipage}%
}
\newcommand{\ntwocolss}[7]{\nntwocolss{#1}{#2}{\includegraphics[#4]{#3}}{#5}{#6}{#7}}

\newcommand{\ntwocolstwofigs}[8]{%
\begin{minipage}[t][\pageheight]{#1}\vspace*{0ex}{#8}\end{minipage}%
\hfill
\begin{minipage}[t]{#2}%
\begin{center}\raisebox{-\height+0.7\baselineskip - #7}{%
\ \hfill\includegraphics[#4]{#3}
}\end{center}\ \vfill\ \begin{center}\raisebox{0in}{%
\ \hfill\includegraphics[#6]{#5}
}\end{center}
\end{minipage}%
}

\newcommand{\ntwocolsstwofigs}[8]{%
\begin{minipage}[t]{#2}%
\begin{center}\raisebox{-\height+0.7\baselineskip - #7}{%
\includegraphics[#4]{#3}
\hfill\ %
}\end{center}%
\ \vfill\ %
\begin{center}\raisebox{0in}{%
\includegraphics[#6]{#5}
\hfill\ %
}\end{center}%
\end{minipage}%
%}
\hfill%
\begin{minipage}[t][\pageheight]{#1}%
\vspace*{0ex}
{
#8%
}%
\end{minipage}%
}

\newcommand\twocolsnormalsize[6]{\ntwocols{#1}{#2}{#3}{#4}{#5}{}{#6}}

\newcommand\twocols[6]{\twocolsnormalsize{#1}{#2}{#3}{#4}{#5}{\small #6}}

\newcommand{\twocolssnormalsize}[6]{\ntwocolss{#1}{#2}{#3}{#4}{#5}{}{#6}}
\newcommand{\twocolss}[6]{\twocolssnormalsize{#1}{#2}{#3}{#4}{#5}{\small #6}}


}{}
%%%%%%%%%%%%%%%%%%%%%%%%%%%%%%%%%%%%%%%%%%%%%%%%%%%%%%%%%%%%%%%%%%%%%%%%%%%%%%%%



%% PACKAGE IMPORTS

\yesnoexec{\pdfpsfrag}{%
	\usepackage{psfrag}%
	\yesnoexec{\reusepsfragpdf}{\usepackage[off]{auto-pst-pdf}}{\usepackage{auto-pst-pdf}}%
}{%
	\usepackage{graphicx}%
}%


\def\trig{no}
\def\probstat{no}
\def\mathfontexamples{no}
\def\preamble{yes}
\def\stories{yes}
\ifthenelse{\boolean{true}}{
\def\algebra{yes}
\def\aalgebra{yes}
\def\meastheory{yes}
\def\topspaces{yes}
\def\absmeas{yes}
\def\measprob{yes}
\def\cvxopt{yes}
\def\sproof{yes}
\def\ynindex{yes}
}{
\def\preamble{no}
\def\trig{no}
\def\stories{no}
\def\algebra{no}
\def\aalgebra{yes}
\def\meastheory{no}
\def\topspaces{no}
\def\absmeas{no}
\def\measprob{no}
\def\cvxopt{no}
\def\sproof{yes}
\def\ynindex{yes}
}
\def\titlepostfix{}
\def\ranalysis{no}
\newcounter{numsectionsforwhichproofexists}
\hypersetup{
colorlinks = true,
linkcolor = [rgb]{0, .5, 1.},
linkbordercolor = {white},
}
\usepackage[normalem]{ulem}
\usepackage{mathtools}
\usepackage{algpseudocode}
\newcommand\bitem{\item [$\because$]}
\newcommand\normal{\mathcalfont{N}}
\newcommand\coll{\collk{C}}
\newcommand\collB{\collk{B}}
\newcommand\collF{\collk{F}}
\newcommand\collG{\collk{G}}
\newcommand\openconv{\collk{U}}
\newcommand{\alg}{\algk{A}}
\newcommand{\algA}{\algk{A}}
\newcommand{\algB}{\algk{B}}
\newcommand{\algC}{\algk{C}}
\newcommand{\algF}{\algk{F}}
\newcommand{\algR}{\algk{R}}
\newcommand{\algX}{\algk{X}}
\newcommand{\algY}{\algk{Y}}
\newcommand{\tXJ}{\topos{X}{J}}
\newcommand{\tJ}{\topol{J}}
\newcommand{\tS}{\topol{S}}
\newcommand\funpageref[1]{-~\pageref{#1}}
\newcommand\funhyperrefnotstarted[2]{\ifthenelse{\equal{#2}{}}{
\textcolor{tiffanyblue}{#1}
}{
\textcolor{tiffanyblue}{#1}~\funpageref{#2}
}}
\newcommand\funhyperref[2]{
\hyperref[#2]{#1}~\funpageref{#2}
}
\newcommand\tocpageref[1]{-~\pageref{#1}}
\newcommand\tochyperref[2]{
\hyperref[#2]{#1}~\tocpageref{#2}
}
\newcommand\refertocounterpartmessage[2]{\vitem [] \hspace{-1.0em}(refer to page~\pageref{#2}\ for {#1} counterpart)}
\newcommand\refertocounterpart[2]{
\ifthenelse{
\equal{#1}{abstract}
\or
\equal{#1}{Lebesgue}
\or
\equal{#1}{normed spaces}
\or
\equal{#1}{complete measure spaces}
}{
\refertocounterpartmessage{#1}{#2}
}{
\errmessage{{#1} should be either 'abstract' or 'Lebesgue' or 'normed spaces' or 'complete measure spaces'}
}
}
\def\talkdate{\today}
\def\talktitle{\hyperref[page:toc]{Searching for Universal Truths}}
\MyLogo{\talktitle}
\rightfooter{\quad\textsf{\thepage}}
\leftheader{Sunghee Yun}
\rightheader{\talkdate}
\newcommand\diagtrirud[6]{
\begin{array}{lcr}
{#1}& \overset{#2}{\longrightarrow} &{#3}
\\%
{}_{#6}\searrow&&\nearrow{}_{#4}
\\%
&{#5}&
\end{array}
}
\newcommand\diagtriruu[6]{
\begin{array}{lcr}
{#1}& \overset{#2}{\longrightarrow} &{#3}
\\%
{}_{#6}\nwarrow&&\nearrow{}_{#4}
\\%
&{#5}&
\end{array}
}
\newcommand\diagsquare[8]{
\begin{array}{rccl}
{#1}&\overset{#2}{\longrightarrow}&{#3}\ \ \ \
\\%
{}_{#8}\uparrow&&\uparrow{}_{#4}
\\%
{#7}&\underset{#6}{\longrightarrow}&{#5}\ \ \ \
\end{array}
}
\begin{document}
\idxtodo{DONE - 2025 0414 - 1 - change mathematicians' names}
\idxtodo{1 - convert bullet points to proper theorem, definition, lemma, corollary, proposition, etc.}
\idxtodo{0 - apply new comma conventions}
\idxtodo{DONE - 2024 0324 - python script for converting slides to doc}
\idxtodo{DONE - 2024 0324 - python script extracting theorem-like list $\to$ using ``list of theorem'' functionality on doc}
\idxtodo{DONE - 2024 0324 - python script extracting figure list $\to$ using ``list of figures'' functionality on doc}
\idxtodo{CANCELED - $<$ 2024 0421 - python script extracting important list}
\idxtodo{DONE - 2024 0324 - change tocpageref and funpageref to hyperlink}
\myfoilhead{\LARGE \talktitle\notempty{\titlepostfix}{\\\LARGE\titlepostfix}{}}
\thispagestyle{empty}
\addtocounter{page}{-1}
\vfill
\vfill
\vfill
{\centering
{\large \bf Sunghee Yun}\\
sunghee.yun@gmail.com\\
}
\vfill
\myfoilhead{Navigating Mathematical and Statistical Territories}
\pagelabel{page:toc}
\bit
\item
Notations \& definitions \& conventions
\bit
\item \tochyperref{notations}
{page:Notations}
\isp
\tochyperref{some definitions}
{page:Some definitions}
\isp
\tochyperref{some conventions}
{page:Some conventions}
\eit
\item
\funhyperref{Algebra}{super-title-page:algebra}
\bit
\item
\funhyperref{inequalities}{title-page:Inequalities}
\isp \funhyperref{number theory}{title-page:number-theory}
\isp \funhyperref{trigonometric functions}{title-page:Trigonometric-functions}
\eit
\newcommand{\itemranalysis}{
\item
\funhyperref{Real analysis}{super-title-page:Real-Analysis}
}
\newcommand{\subitemsmeastheory}{
\item
\funhyperref{Lebesgue measure}{title-page:lebesgue-measure}
\isp \funhyperref{Lebesgue measurable functions}{title-page:measurable-functions}
\isp \funhyperref{Lebesgue integral}{title-page:lebesgue-integral}
}
\newcommand{\subitemstopspaces}{
\item
\funhyperref{space overview}{title-page:Space Overview}
\isp \funhyperref{classical Banach spaces}{title-page:classical-banach-spaces}
\item
\funhyperref{metric spaces}{title-page:metric-spaces}
\isp \funhyperref{topological spaces}{title-page:topological-spaces}
\isp \funhyperref{compact and locally compact spaces}{title-page:Compact-and-Locally-Compact-Spaces}
\isp \funhyperref{Banach spaces}{title-page:Banach-Spaces}
}
\newcommand{\subitemsgenmeas}{
\item
\funhyperref{measure and integration}{title-page:Measure-and-Integration}
\isp \funhyperref{measure and outer measure}{title-page:Measure and Outer Measure}
}
\item
Proof \& references \& indices
\bit
\item
\ifthenelse{\equal{\value{numsectionsforwhichproofexists}}{0} \OR \equal{\sproof}{no}}{}{
\tochyperref{selected proofs}
{super-title-page:Proofs}
\isp
}
\tochyperref{references}
{super-title-page:References}
\eit
\eit
\myfoilhead{Notations}
\pagelabel{page:Notations}
\bit
\item
sets of numbers
\bit
\item
$\naturals$ - set of natural numbers
\index{natural number}
\index{number!natural number}
\item
$\integers$ - set of integers
\index{integer}
\index{number!integer}
\item
$\integers_+$ - set of nonnegative integers
\item
$\rationals$ - set of rational numbers
\index{rational number}
\index{number!rational number}
\item
$\reals$ - set of real numbers
\index{real number}
\index{number!real number}
\item
$\preals$ - set of nonnegative real numbers
\item
$\ppreals$ - set of positive real numbers
\item
$\complexes$ - set of complex numbers
\index{complex number}
\index{number!complex number}
\eit
\item
sequences $\seq{x_i}$ and the like
\index{sequence}
\bit
\item
finite $\seq{x_i}_{i=1}^n$, infinite $\seq{x_i}_{i=1}^\infty$ - use $\seq{x_i}$ whenever unambiguously understood
\index{finite sequence}
\index{sequence!finite sequence}
\index{infinite sequence}
\index{sequence!infinite sequence}
\item
similarly for other operations, \eg, $\sum x_i$, $\prod x_i$, $\cup A_i$, $\cap A_i$, $\bigtimes A_i$
\item
similarly for integrals, \eg, $\int f$ for $\int_{-\infty}^\infty f$
\eit
\item
sets
\bit
\item
$\compl{A}$ - complement of $A$
\index{complement!set}
\index{set!complement}
\item
$A\sim B$ - $A\cap \compl{B}$
\index{difference!set}
\index{set!difference}
\item
$A\Delta B$ - $(A\cap \compl{B}) \cup (\compl{A} \cap B)$
\item
$\powerset(A)$ - set of all subsets of $A$
\eit
\item
sets in metric vector spaces
\bit
\item
\closure{A} - closure of set $A$
\index{closure!set}
\index{set!closure}
\item
\interior{A} - interior of set $A$
\index{interior!set}
\index{set!interior}
\item
$\relint A$ - relative interior of set $A$
\index{relative interior!set}
\index{set!relative interior}
\item
$\boundary A$ - boundary of set $A$
\index{boundary!set}
\index{set!boundary}
\eit
\item
set algebra
\bit
\item
$\sigma(\subsetset{A})$ - $\sigma$-algebra generated by \subsetset{A},
\ie, smallest $\sigma$-algebra containing \subsetset{A}
\index{smallest $\sigma$-algebra containing subsets}
\eit
\item
norms in $\reals^n$
\index{vector!norm}
\index{norm!vector}
\bit
\item
$\|x\|_p$ ($p\geq1$) - $p$-norm of $x\in\reals^n$, \ie, $(|x_1|^p + \cdots + |x_n|^p)^{1/p}$
\item
\eg, $\|x\|_2$ - Euclidean norm
\eit
\item
matrices and vectors
\bit
\item $a_{i}$ - $i$-th entry of vector $a$
\item $A_{ij}$ - entry of matrix $A$ at position $(i,j)$,
\ie, entry in $i$-th row and $j$-th column
\item $\Tr(A)$ - trace of $A \in\reals^{n\times n}$,
\ie, $A_{1,1}+ \cdots + A_{n,n}$
\index{matrix!trace}
\index{trace!matrix}
\eit
\item
symmetric, positive definite, and positive semi-definite matrices
\bit
\item
$\symset{n}\subset \reals^{n\times n}$ - set of symmetric matrices
\index{symmetric matrix}
\index{matrix!symmetric}
\item
$\possemidefset{n}\subset \symset{n}$ - set of positive semi-definite matrices;
$A\succeq0 \Leftrightarrow A \in \possemidefset{n}$
\index{positive semi-definite matrix}
\index{matrix!positive semi-definite}
\item
$\posdefset{n}\subset \symset{n}$ - set of positive definite matrices;
$A\succ0 \Leftrightarrow A \in \posdefset{n}$
\index{positive definite matrix}
\index{matrix!positive definite}
\eit
\item
sometimes,
use Python script-like notations
(with serious abuse of mathematical notations)
\bit
\item
use $f:\reals\to\reals$ as if it were $f:\reals^n \to \reals^n$,
\eg,
$$
\exp(x) = (\exp(x_1), \ldots, \exp(x_n)) \quad \mbox{for } x\in\reals^n
$$
and
$$
\log(x) = (\log(x_1), \ldots, \log(x_n)) \quad \mbox{for } x\in\ppreals^n
$$
which corresponds to Python code {\tt numpy.exp(x)} or {\tt numpy.log(x)}
where {\tt x} is instance of {\tt numpy.ndarray}, \ie, {\tt numpy} array
\item
use $\sum x$ to mean $\ones^T x$ for $x\in\reals^n$,
\ie\
$$
\sum x = x_1 + \cdots + x_n
$$
which corresponds to Python code {\tt x.sum()}
where {\tt x} is {\tt numpy} array
\item
use $x/y$ for $x,y\in\reals^n$ to mean
$$
\rowvecthree{x_1/y_1}{\cdots}{x_n/y_n}^T
$$
which corresponds to Python code {\tt x / y}
where {\tt x} and {\tt y} are $1$-d {\tt numpy} arrays
\item
use $X/Y$ for $X,Y\in\reals^{m\times n}$ to mean
$$
\begin{my-matrix}{cccc}
X_{1,1}/Y_{1,1} & X_{1,2}/Y_{1,2} & \cdots & X_{1,n}/Y_{1,n}
\\
X_{2,1}/Y_{2,1} & X_{2,2}/Y_{2,2} & \cdots & X_{2,n}/Y_{2,n}
\\
\vdots & \vdots & \ddots & \vdots
\\
X_{m,1}/Y_{m,1} & X_{m,2}/Y_{m,2} & \cdots & X_{m,n}/Y_{m,n}
\end{my-matrix}
$$
which corresponds to Python code {\tt X / Y}
where {\tt X} and {\tt Y} are $2$-d {\tt numpy} arrays
\eit
\eit
\vfill
\myfoilhead{Some definitions}
\pagelabel{page:Some definitions}
\begin{mydefinition}{infinitely often - i.o.}
\index{infinitely often}
\index{i.o.!infinitely often}
statement $P_n$, said to happen \define{infinitely often} or \define{i.o.} if
$$
\left(
\forall N\in\naturals
\right)
\left(
\exists n > N
\right)
\left(
P_n
\right)
$$
\end{mydefinition}
\vfill
\begin{mydefinition}{almost everywhere - a.e.}
\index{almost everywhere}
\index{a.e.!almost everywhere}
\index{almost surely}
\index{a.s.!almost surely}
statement $P(x)$,
said to happen \define{almost everywhere} or \define{a.e.} or \define{almost surely} or \define{a.s.}\
(depending on context)
associated with
measure space \meas{X}{\algB}{\mu}\
if
$$
\mu \set{x}{P(x)} = 1
$$
or equivalently
$$
\mu \set{x}{\sim P(x)} = 0
$$
\end{mydefinition}
\vfill
\myfoilhead{Some conventions}
\pagelabel{page:Some conventions}
\bit
\item
(for some subjects) use following conventions
\bit
\vitem
$0\cdot \infty = \infty \cdot 0 = 0$
\vitem
$(\forall x\in\ppreals)(x\cdot \infty = \infty \cdot x = \infty)$
\vitem
$\infty \cdot \infty = \infty$
\eit
\eit
\vfillfi
\TITLEFOIL{Algebra}{algebra}
\nocite{HLP:52}
\titlefoil{Inequalities}{Inequalities}
\myfoilhead{Jensen's inequality}
\bit
\item
strictly convex function: for any $x\neq y$ and $0< \alpha <1$
\index{convex functions}
\index{convex functions!strictly}
(\definitionname~\ref{definition:convex functions})
\[
\alpha f(x) + (1-\alpha) f(y) > f(\alpha x + (1-\alpha) y)
\]
\item convex function: for any $x, y$ and $0< \alpha <1$
\index{convex functions}
(\definitionname~\ref{definition:convex functions})
\[
\alpha f(x) + (1-\alpha) f(y) \geq f(\alpha x + (1-\alpha) y)
\]
\eit
\begin{myinequality}{Jensen's inequality - for finite sequences}
\index{Jensen's inequality!for finite sequences}
\index{Jensen, Johan Ludwig William Valdemar!Jensen's inequality!for finite sequences}
for convex function $f$ and \emph{distinct} $x_i$
and $0< \alpha_i <1$ with $\alpha_1 + \cdots = \alpha_n=1$
\index{Jensen's inequality}
\index{inequalities!Jensen's inequality}
\index{Jensen, Johan Ludwig William Valdemar!inequality}
\[
\alpha_1 f(x_1) + \cdots + \alpha_n f(x_n) \geq f(\alpha_1 x_1 + \cdots + \alpha_n x_n)
\]
\bit
\item
if $f$ is strictly convex, equality holds \iaoi\ $x_1=\cdots=x_n$
\eit
\end{myinequality}
\myfoilhead{Jensen's inequality - for random variables}
\bit
\item
discrete random variable interpretation of Jensen's inequality in summation form - assume $\Prob(X=x_i) = \alpha_i$, then
\[
\Expect f(X)
=
\alpha_1 f(x_1) + \cdots + \alpha_n f(x_n)
\geq
f(\alpha_1 x_1 + \cdots + \alpha_n x_n)
=
f\left(\Expect X\right)
\]
\vitem true for any random variables $X$
\eit
\vfill
\begin{myinequality}{Jensen's inequality - for random variables}
\index{Jensen's inequality!for random variables}
\index{Jensen, Johan Ludwig William Valdemar!Jensen's inequality!for random variables}
for random vector $X$ (page~\pageref{page:random-variables} for definition)
\[
\Expect f(X) \geq f(\Expect X)
\]
if probability density function (PDF) $p_X$ given,
\[
\int f(x) p_X(x) dx \geq f\left(\int x p_X(x) dx\right)
\]
\end{myinequality}
\vfill
\myfoilhead{Proof for $n=3$}
\bit
\item for any $x,y,z$ and $\alpha,\beta,\gamma>0$ with $\alpha + \beta + \gamma = 1$
\begin{eqnarray*}
\alpha f(x) + \beta f(y) + \gamma f(z)
&=&
(\alpha+\beta)\left(\frac{\alpha}{\alpha+\beta} f(x) + \frac{\beta}{\alpha + \beta} f(y)\right) + \gamma f(z)
\\
&\geq&
(\alpha+\beta)f\left(\frac{\alpha}{\alpha+\beta} x + \frac{\beta}{\alpha + \beta} y\right) + \gamma f(z)
\\
&\geq&
f\left((\alpha+\beta)\left(\frac{\alpha}{\alpha+\beta} x + \frac{\beta}{\alpha + \beta} y\right) + \gamma z \right)
\\
&=&
f(\alpha x + \beta y + \gamma z )
\end{eqnarray*}
\eit
\myfoilhead{Proof for all $n$}
\bit
\item
use mathematical induction
\bit
\vitem
assume that Jensen's inequality holds for $1\leq n\leq m$
\vitem
for distinct $x_i$ and $\alpha_i>0$ ($1\leq i\leq m+1$) with $\alpha_1 + \cdots + \alpha_{m+1} = 1$
\begin{eqnarray*}
\sum^{m+1}_{i=1} \alpha_i f(x_i)
&=&
\left(\sum^m_{j=1} \alpha_j\right) \sum^m_{i=1} \left(\frac{\alpha_i}{\sum^m_{j=1} \alpha_j} f(x_i)\right) + \alpha_{m+1} f(x_{m+1})
\\
&\geq&
\left(\sum^m_{j=1} \alpha_j\right) f\left(\sum^m_{i=1} \left(\frac{\alpha_i}{\sum^m_{j=1} \alpha_j} x_i\right)\right) + \alpha_{m+1} f(x_{m+1})
\\
&=&
\left(\sum^m_{j=1} \alpha_j\right) f\left(\frac{1}{\sum^m_{j=1} \alpha_j}\sum^m_{i=1} {\alpha_i}{} x_i\right) + \alpha_{m+1} f(x_{m+1})
\\
&\geq&
f\left( \sum^m_{i=1} \alpha_i x_i + \alpha_{m+1} x_{m+1}\right)
=
f\left( \sum^{m+1}_{i=1} \alpha_i x_i \right)
\end{eqnarray*}
\eit
\eit
\myfoilhead{1st and 2nd order conditions for convexity}
\bit
\item 1st order condition (assuming differentiable $f:\reals\to\reals$)
- $f$ is strictly convex \iaoi\ for any $x\neq y$
\index{convex functions!first order condition}
\[
f(y) > f(x) + f'(x)(y-x)
\]
\vitem 2nd order condition (assuming twice-differentiable $f:\reals\to\reals$)
\index{convex functions!second order condition}
\bit
\vitem if $f''(x)>0$, $f$ is strictly convex
\vitem $f$ is convex \iaoi\ for any $x$
\[
f''(x) \geq 0
\]
\eit
\eit
\vfill
\myfoilhead{Jensen's inequality examples}
\bit
\item $f(x)=x^2$ is strictly convex
\[
\frac{a^2 + b^2}{2}
\geq
\left(\frac{a+b}{2}\right)^2
\]
\vitem $f(x)=x^4$ is strictly convex
\[
\frac{a^4 + b^4}{2}
\geq
\left(\frac{a+b}{2}\right)^4
\]
\vitem $f(x)=\exp(x)$ is strictly convex
\[
\frac{\exp(a) + \exp(b)}{2}
\geq
\exp\left(\frac{a+b}{2}\right)
\]
\vvitem equality holds \iaoi\ $a=b$ for all inequalities
\eit
\myfoilhead{1st and 2nd order conditions for convexity - vector version}
\bit
\vitem
1st order condition (assuming differentiable $f:\reals^n\to\reals$)
- $f$ is strict convex \iaoi\ for any $x,y$
\index{convex functions!first order condition!vector functions}
\[
f(y) > f(x) + \nabla f(x)^T (y-x)
\]
where $\nabla f(x) \in\reals^{n}$ with $\nabla f(x)_{i} = \partial f(x) / \partial x_i$
\vitem
2nd order condition (assuming twice-differentiable $f:\reals^n\to\reals$)
\index{convex functions!second order condition!vector functions}
\bit
\vitem
if $\nabla^2 f(x) \succ 0$, $f$ is strictly convex
\vitem
$f$ is convex \iaoi\ for any $x$
\[
\nabla^2 f(x)\succeq 0
\]
\eit
where $\nabla^2 f(x) \in\reals^{n\times n}$
is Hessian matrix of $f$ evaluated at $x$,
\ie,
$\nabla^2 f(x)_{i,j} = \partial^2 f(x) / \partial x_i \partial x_j$
\eit
\vfill
\myfoilhead{Jensen's inequality examples - vector version}
\bit
\item assume $f:\reals^n\to\reals$
\vitem $f(x)=\|x\|_2 = \sqrt{\sum x_i^2}$ is strictly convex
\[
(\|a\|_2 + 2\|b\|_2 )/3
\geq
\left\|(a+2b)/3\right\|_2
\]
\bit
\item equality holds \iaoi\ $a=b\in\reals^n$
\eit
\vitem $f(x)=\|x\|_p = \left(\sum |x_i|^p\right)^{1/p}$ ($p>1$) is strictly convex
\[
\frac{1}{k}
\left(\sum_{i=1}^k\|x^{(i)}\|_p \right)
\geq
\left\|\frac{1}{k}\sum_{i=1}^k x^{(i)}\right\|_p
\]
\bit
\item equality holds \iaoi\ $x^{(1)}=\cdots=x^{(k)}\in\reals^n$
\eit
\eit
\myfoilhead{AM $\geq$ GM}
\bit
\item
for all $a,b>0$
\[
\frac{a + b}{2} \geq \sqrt{ab}
\]
\bit
\item
equality holds if and only if $a=b$
\eit
\vitem
below most general form holds
\eit
\begin{myinequality}{AM-GM inequality}
for any $n$ $a_i>0$ and $\alpha_i>0$ with $\alpha_1+\cdots+\alpha_n=1$
\[
\alpha_1 a_1 + \cdots + \alpha_n a_n
\geq
{a_1^{\alpha_1} \cdots a_n^{\alpha_n}}
\]
where equality holds if and only if $a_1=\cdots=a_n$
\end{myinequality}
\bit
\item
let's prove these incrementally
(for rational $\alpha_i$)
\eit
\vfill
\myfoilhead{Proof of AM $\geq$ GM - simplest case}
\bit
\item use fact that $x^2\geq0$ for any $x\in\reals$
\vitem for any $a,b>0$
\begin{eqnarray*}
&&
(\sqrt{a}-\sqrt{b})^2 \geq 0
\\
&\Leftrightarrow&
a^2 - 2\sqrt{ab} + b^2 \geq 0
\\
&\Leftrightarrow&
a + b \geq 2\sqrt{ab}
\\
&\Leftrightarrow&
\frac{a + b}{2} \geq \sqrt{ab}
\end{eqnarray*}
\bit
\item equality holds if and only if $a=b$
\eit
\eit
\myfoilhead{Proof of AM $\geq$ GM - when $n=4$ and $n=8$}
\bit
\item for any $a,b,c,d>0$
\[
\frac{a+b+c+d}{4}
\geq
\frac{2\sqrt{ab} + 2\sqrt{cd}}{4}
=
\frac{\sqrt{ab} + \sqrt{cd}}{2}
\geq
\sqrt{\sqrt{ab} \sqrt{cd}}
=
\sqrt[4]{abcd}
\]
\bit
\item equality holds if and only if $a=b$ and $c=d$ and $ab=cd$
if and only if $a=b=c=d$
\eit
\vitem likewise, for $a_1,\ldots,a_8>0$
\begin{eqnarray*}
\frac{a_1+\cdots+a_8}{8}
&\geq&
\frac{\sqrt{a_1a_2} + \sqrt{a_3a_4} + \sqrt{a_5a_6} + \sqrt{a_7a_8}}{4}
\\
&\geq&
\sqrt[4]{\sqrt{a_1a_2} \sqrt{a_3a_4} \sqrt{a_5a_6} \sqrt{a_7a_8}}
\\
&=&
\sqrt[8]{a_1\cdots a_8}
\end{eqnarray*}
\bit
\item equality holds if and only if $a_1=\cdots=a_8$
\eit
\eit
\myfoilhead{Proof of AM $\geq$ GM - when $n=2^m$}
\bit
\item generalized to cases $n=2^m$
\[
\left(\sum_{a=1}^{2^m} a_i\right) / 2^m\geq \left({\prod_{a=1}^{2^m} a_i}\right)^{1/2^m}
\]
\bit
\item equality holds if and only if $a_1=\cdots=a_{2^m}$
\eit
\vitem can be proved by \emph{mathematical induction}
\eit
\myfoilhead{Proof of AM $\geq$ GM - when $n=3$}
\bit
\item proof for $n=3$
\begin{eqnarray*}
&&
\frac{a+b+c}{3} = \frac{a + b + c + (a+b+c)/3}{4}
\geq \sqrt[4]{abc(a+b+c)/3}
\\
&\Rightarrow&
\left(\frac{a+b+c}{3}\right)^4 \geq {abc(a+b+c)/3}
\\
&\Leftrightarrow&
\left(\frac{a+b+c}{3}\right)^3 \geq abc
\\
&\Leftrightarrow&
\frac{a+b+c}{3} \geq \sqrt[3]{abc}
\end{eqnarray*}
\bit
\item equality holds if and only if $a=b=c=(a+b+c)/3$ if and only if $a=b=c$
\eit
\eit
\myfoilhead{Proof of AM $\geq$ GM - for all integers}
\bit
\item for any integer $n\neq 2^m$
\vitem for $m$ such that $2^m>n$
\begin{eqnarray*}
&&
\frac{a_1+\cdots+a_n}{n} = \frac{a_1 + \cdots + a_n + (2^m-n) (a_1+\cdots+a_n) /n}{2^m}
\\
&&
\geq
\sqrt[2^m]{a_1\cdots a_n \cdot ((a_1 + \cdots + a_n)/n)^{2^m-n}}
\\
&\Leftrightarrow&
\left(\frac{a_1+\cdots+a_n}{n}\right)^{2^m}
\geq
{a_1\cdots a_n \cdot \left(\frac{a_1 + \cdots + a_n}{n}\right)^{2^m-n}}
\\
&\Leftrightarrow&
\left(\frac{a_1+\cdots+a_n}{n}\right)^{n}
\geq
{a_1\cdots a_n}
\\
&\Leftrightarrow&
\frac{a_1+\cdots+a_n}{n}
\geq
\sqrt[n]{a_1\cdots a_n}
\end{eqnarray*}
\bit
\item equality holds if and only if $a_1=\cdots=a_n$
\eit
\eit
\myfoilhead{Proof of AM $\geq$ GM - rational $\alpha_i$}
\bit
\item
given $n$ positive rational $\alpha_i$,
we can find $n$ natural numbers $q_i$
such that
\[
\alpha_i = \frac{q_i}{ N}
\]
where $q_1+\cdots+q_n=N$
\vitem
for any $n$ positive $a_i\in\reals$ and positive $n$ $\alpha_i\in\rationals$ with $\alpha_1+\cdots+\alpha_n=1$
\[
\alpha_1 a_1 + \cdots + \alpha_n a_n
= \frac{q_1 a_1 + \cdots + q_n a_n}{N}
\geq \sqrt[N]{a_1^{q_1}\cdots a_n^{q_n}}
= a_1^{\alpha_1}\cdots a_n^{\alpha_n}
\]
\bit
\item
equality holds if and only if $a_1=\cdots=a_n$
\eit
\eit
\myfoilhead{Proof of AM $\geq$ GM - real $\alpha_i$}
\bit
\item
exist $n$ rational sequences $\{ \beta_{i,1}, \beta_{i,2}, \ldots\}$ ($1\leq i\leq n$) such that
\begin{eqnarray*}
&&
\beta_{1,j}+\cdots+\beta_{n,j}=1 \ \forall \ j\geq1
\\
&&
\lim_{j\to\infty} \beta_{i,j} = \alpha_i \ \forall \ 1\leq i\leq n
\end{eqnarray*}
\vitem
for all $j$
\[
\beta_{1,j} a_1 + \cdots + \beta_{n,j} a_n
\geq
a_1^{\beta_{1,j}}\cdots a_n^{\beta_{n,j}}
\]
hence
\begin{eqnarray*}
&&
\lim_{j\to\infty} \left(\beta_{1,j} a_1 + \cdots + \beta_{n,j} a_n \right)
\geq
\lim_{j\to\infty} a_1^{\beta_{1,j}}\cdots a_n^{\beta_{n,j}}
\\
&\Leftrightarrow&
\alpha_1 a_1 + \cdots + \alpha_n a_n
\geq
a_1^{\alpha_1}\cdots a_n^{\alpha_n}
\end{eqnarray*}
\vitem \emph{cannot} prove equality condition from above proof method
\eit
\myfoilhead{Proof of AM $\geq$ GM using Jensen's inequality}
\bit
\item
$(-\log)$ is strictly convex function because
\[
\frac{d^2}{dx^2} \left(-\log(x)\right)
= \frac{d}{dx} \left(-\frac{1}{x} \right)
= \frac{1}{x^2} > 0
\]
\vitem
Jensen's inequality implies for $a_i >0$, $\alpha_i >0$ with $\sum \alpha_i = 1$
\begin{eqnarray*}
-\log\left(\prod a_i^{\alpha_i}\right)
= -\sum \log\left( a_i^{\alpha_i}\right)
=
\sum \alpha_i (-\log(a_i)) \geq -\log \left(\sum \alpha_i a_i\right)
\end{eqnarray*}
\vitem
$(-\log)$ strictly monotonically decreases, hence $\prod a_i^{\alpha_i} \leq \sum \alpha_i a_i$,
having just proved
\[
\alpha_1 a_1 + \cdots + \alpha_n a_n
\geq
a_1^{\alpha_1}\cdots a_n^{\alpha_n}
\]
where equality if and only if $a_i$ are equal
(by Jensen's inequality's equality condition)
\eit
\myfoilhead{Cauchy-Schwarz inequality}
\begin{myinequality}{Cauchy-Schwarz inequality}
for any $a_i, b_i\in\reals$
\index{Cauchy-Schwarz inequality}
\index{Schwarz, Hermann!Cauchy-Schwarz inequality}
\index{Cauchy, Augustin-Louis!Cauchy-Schwarz inequality}
\index{Schwarz, Hermann!Cauchy-Schwarz inequality}
\[
( a_1^2 + \cdots + a_n^2 )
( b_1^2 + \cdots + b_n^2 )
\geq
(a_1b_1 + \cdots + a_nb_n)^2
\]
\end{myinequality}
\bit
\vitem middle school proof
\begin{eqnarray*}
&&\sum (t a_i + b_i)^2 \geq 0 \ \forall\ t \in \reals
\\
&\Leftrightarrow&
t^2 \sum a_i^2 + 2t \sum a_ib_i + \sum b_i^2 \geq 0 \ \forall\ t \in \reals
\\
&\Leftrightarrow&
\Delta = \left(\sum a_ib_i \right)^2 - \sum a_i^2 \sum b_i^2 \leq 0
\end{eqnarray*}
\bit
\item equality holds if and only if $\exists t\in\reals$, $t a_i + b_i=0$ for all $1\leq i\leq n$
\eit
\eit
\myfoilhead{Cauchy-Schwarz inequality - another proof}
\bit
\item $x\geq0$ for any $x\in\reals$, hence
\begin{eqnarray*}
&&
\sum_i \sum_j (a_ib_j - a_jb_i)^2 \geq0
\\
&\Leftrightarrow&
\sum_i \sum_j (a_i^2b_j^2 - 2a_ia_jb_ib_j + a_j^2b_i^2) \geq0
\\
&\Leftrightarrow&
\sum_i \sum_j a_i^2b_j^2 + \sum_i \sum_j a_j^2b_i^2 -2 \sum_i \sum_j a_ia_jb_ib_j \geq 0
\\
&\Leftrightarrow&
2 \sum_i a_i^2 \sum_j b_j^2 - 2 \sum_i a_ib_i \sum_j a_jb_j \geq 0
\\
&\Leftrightarrow&
\sum_i a_i^2 \sum_j b_j^2 - \left(\sum_i a_ib_i\right)^2 \geq0
\end{eqnarray*}
\bit
\item equality holds if and only if $a_ib_j=a_jb_i$ for all $1\leq i,j\leq n$
\eit
\eit
\myfoilhead{Cauchy-Schwarz inequality - still another proof}
\bit
\item for any $x,y\in\reals$ and $\alpha,\beta>0$ with $\alpha + \beta = 1$
\begin{eqnarray*}
&&
(\alpha x - \beta y)^2
=
\alpha^2 x^2 + \beta^2 y^2 - 2\alpha \beta xy
\\
&&
=
\alpha(1-\beta) x^2 + (1-\alpha)\beta y^2 - 2\alpha \beta xy
\geq
0
\\
&\Leftrightarrow&
\alpha x^2 + \beta y^2
\geq
\alpha \beta x^2 + \alpha \beta y^2 + 2\alpha \beta xy
= \alpha \beta (x+y)^2
\\
&\Leftrightarrow&
x^2 / \alpha + y^2 / \beta \geq (x+y)^2
\end{eqnarray*}
\item plug in $x=a_i$, $y=b_i$, $\alpha = A/(A+B)$, $\beta=B/(A+B)$
where $A = \sqrt{\sum a_i^2}$, $B = \sqrt{\sum b_i^2}$
\begin{eqnarray*}
&&
\sum (a_i^2 / \alpha + b_i^2 / \beta) \geq \sum (a_i+b_i)^2
\Leftrightarrow
(A+B)^2 \geq A^2 + B^2 + 2 \sum a_i b_i
\\
&\Leftrightarrow&
AB \geq \sum a_i b_i
\Leftrightarrow
A^2B^2 \geq \left(\sum a_i b_i\right)^2
\Leftrightarrow
{\sum a_i^2}{\sum b_i^2} \geq \left(\sum a_i b_i \right)^2
\end{eqnarray*}
\eit
\myfoilhead{Cauchy-Schwarz inequality - proof using determinant}
\bit
\item almost the same proof as first one - but using $2$-by-$2$ matrix determinant
\begin{eqnarray*}
&&\sum (x a_i + y b_i )^2 \geq 0 \ \forall\ x,y \in \reals
\\
&\Leftrightarrow&
x^2 \sum a_i^2 + 2xy \sum a_ib_i + y^2\sum b_i^2 \geq 0 \ \forall \ x, y \in \reals
\\
&\Leftrightarrow&
\begin{my-matrix}{cc}
x & y
\end{my-matrix}
\begin{my-matrix}{cc}
\sum a_i^2 & \sum a_ib_i
\\
\sum a_ib_i & \sum b_i^2
\end{my-matrix}
\begin{my-matrix}{c}
x \\ y
\end{my-matrix}
\geq 0
\ \forall \ x, y \in \reals
\\
\\
&\Leftrightarrow&
\left|
\begin{array}{cc}
\sum a_i^2 & \sum a_ib_i
\\
\sum a_ib_i & \sum b_i^2
\end{array}
\right|
\geq 0
\Leftrightarrow
\sum a_i^2 \sum b_i^2 - \left(\sum a_ib_i \right)^2 \geq0
\end{eqnarray*}
\bit
\item equality holds \iaoi\
$$
\left(
\exists x,y\in\reals \mbox{ with } xy\neq0
\right)
\left(
xa_i + yb_i=0\ \
\forall 1\leq i\leq n
\right)
$$
\eit
\vitem allows \eemph{beautiful generalization} of Cauchy-Schwarz inequality
\eit
\myfoilhead{Cauchy-Schwarz inequality - generalization}
\index{Cauchy-Schwarz inequality!generalization}
\index{Schwarz, Hermann!Cauchy-Schwarz inequality!generalization}
\index{Cauchy, Augustin-Louis!Cauchy-Schwarz inequality!generalization}
\pagelabel{page:Cauchy-Schwarz inequality - generalization}
\bit
\item
want to say something like $\sum_{i=1}^n (x a_i + y b_i + z c_i + w d_i + \cdots)^2$
\vitem run out of alphabets \ldots\ - use double subscripts
\begin{eqna}
&&
\sum_{i=1}^n (x_1 A_{1,i} + x_2 A_{2,i} + \cdots + x_m A_{m,i})^2 \geq 0 \ \forall\ x_i \in \reals
\\
&\Leftrightarrow&
\sum_{i=1}^n (x^T a_i)^2
=
\sum_{i=1}^n x^T a_ia_i^T x
=
x^T \left(\sum_{i=1}^n a_ia_i^T\right) x \geq 0 \ \forall\ x \in \reals^m
\\
&\Leftrightarrow&
\left|
\begin{array}{cccc}
\sum_{i=1}^n A_{1,i}^2 & \sum_{i=1}^n A_{1,i} A_{2,i} & \cdots & \sum_{i=1}^n A_{1,i} A_{m,i}
\\
\sum_{i=1}^n A_{1,i}A_{2,i} & \sum_{i=1}^n A_{2,i}^2 & \cdots & \sum_{i=1}^n A_{2,i} A_{m,i}
\\
\vdots & \vdots & \ddots & \vdots
\\
\sum_{i=1}^n A_{1,i}A_{m,i} & \sum_{i=1}^n A_{2,i}A_{m,i} & \cdots & \sum_{i=1}^n A_{m,i}^2
\end{array}
\right|
\geq 0
\end{eqna}
\vspace{1em}
\bit
\item []
where $a_i = \begin{my-matrix}{ccc} A_{1,i} &\cdots & A_{m,i}\end{my-matrix}^T \in\reals^m$
\vitem
equality holds \iaoi\ $\exists x\neq0\in\reals^m$, $x^Ta_i =0$ for all $1\leq i\leq n$\
\eit
\eit
\myfoilhead{Cauchy-Schwarz inequality - three series of variables}
\bit
\item
let $m=3$
\begin{eqnarray*}
&&
\begin{my-matrix}{ccc}
\sum a_{i}^2 & \sum a_{i} b_{i} & \sum a_{i} c_{i}
\\
\sum a_{i}b_{i} & \sum b_{i}^2 & \sum b_{i} c_{i}
\\
\sum a_{i}c_{i} & \sum b_{i}c_{i} & \sum c_{i}^2
\end{my-matrix}
\succeq 0
\\
&\Rightarrow&
\sum a_i^2 \sum b_i^2 \sum c_i^2 + 2 \sum a_ib_i \sum b_ic_i \sum c_ia_i
\\
&&
\geq \sum a_i^2 \left(\sum b_i c_i\right)^2 + \sum b_i^2 \left(\sum a_i c_i\right)^2 + \sum c_i^2 \left(\sum a_i b_i\right)^2
\end{eqnarray*}
\bit
\item
equality holds if and only if $\exists x,y,z\in\reals$, $xa_i + yb_i + zc_i=0$ for all $1\leq i\leq n$
\eit
\vitem
questions for you
\bit
\vitem
what does this mean?
\vitem
any real-world applications?
\eit
\eit
\myfoilhead{Cauchy-Schwarz inequality - extensions}
\index{Cauchy-Schwarz inequality!extension}
\index{Schwarz, Hermann!Cauchy-Schwarz inequality!extension}
\index{Cauchy, Augustin-Louis!Cauchy-Schwarz inequality!extension}
\begin{myinequality}{Cauchy-Schwarz inequality - for complex numbers}
\index{Cauchy-Schwarz inequality!for complex numbers}
\index{Schwarz, Hermann!Cauchy-Schwarz inequality!for complex numbers}
\index{Cauchy, Augustin-Louis!Cauchy-Schwarz inequality!for complex numbers}
for $a_i, b_i \in\complexes$
\[
\sum |a_i|^2 \sum |b_i|^2 \geq \left|\sum a_i b_i \right|^2
\]
\end{myinequality}
\vspace{-.3cm}
\begin{myinequality}{Cauchy-Schwarz inequality - for infinite sequences}
\index{Cauchy-Schwarz inequality!for infinite sequences}
\index{Schwarz, Hermann!Cauchy-Schwarz inequality!for infinite sequences}
\index{Cauchy, Augustin-Louis!Cauchy-Schwarz inequality!for infinite sequences}
for two complex infinite sequences
$\seq{a_i}_{i=1}^\infty$
and
$\seq{b_i}_{i=1}^\infty$
\[
\sum^\infty_{i=1} |a_i|^2 \sum^\infty_{i=1} |b_i|^2 \geq \left|\sum^\infty_{i=1} a_i b_i \right|^2
\]
\end{myinequality}
\vspace{-.3cm}
\begin{myinequality}{Cauchy-Schwarz inequality - for complex functions}
\index{Cauchy-Schwarz inequality!for complex functions}
\index{Schwarz, Hermann!Cauchy-Schwarz inequality!for complex functions}
\index{Cauchy, Augustin-Louis!Cauchy-Schwarz inequality!for complex functions}
for two complex functions $f,g:[0,1] \to \complexes$
\[
\int |f|^2 \int |g|^2 \geq \left|\int f g \right|^2
\]
\end{myinequality}
\bit
\vitem note that \eemph{all these can be further generalized
as in page~\pageref{page:Cauchy-Schwarz inequality - generalization}}
\eit
\titlefoil{Number Theory - Queen of Mathematics}{number-theory}
\myfoilhead{Integers}
\bit
\item
integers ($\integers$)
-
$\ldots -2, -1, 0, 1, 2, \ldots$
\bit
\vitem
first defined by Bertrand Russell
\vitem
algebraic structure - commutative ring
\bit
\vitem [-]
addition, multiplication defined, but divison \emph{not} defined
\vitem [-]
addition, multiplication are associative
\vitem [-]
multiplication distributive over addition
\vitem [-]
addition, multiplication are commutative
\eit
\eit
\vitem
natural numbers ($\naturals$)
\bit
\vitem
$1, 2, \ldots$
\eit
\eit
\myfoilhead{Division and prime numbers}
\bit
\item divisors for $n\in\naturals$
\[
\set{d\in\naturals}{ d \mbox{ divides } n}
\]
\vitem prime numbers
\bit
\item $p$ is primes if $1$ and $p$ are only divisors
\eit
\eit
\labelfoilhead{Fundamental theorem of arithmetic}
\begin{mytheorem}{fundamental theorem of arithmetic}
\index{fundamental theorem of arithmetic}
\index{fundamental theorem!of arithmetic}
integer $n\geq2$ can be factored uniquely into products of primes,
\ie,
exist distinct primes, $p_1$, \ldots, $p_k$, and $e_1,\ldots, e_k\in\naturals$
such that
$$
n = p_1^{e_1} p_2^{e_2} \cdots p_k^{e_k}
$$
\end{mytheorem}
\bit
\vitem
hence,
integers are \emph{factorial ring}
(\definitionname~\ref{definition:factorial ring})
\eit
\vfill
\myfoilhead{Elementary quantities}
\bit
\item
greatest common divisor (gcd) (of $a$ and $b$)
\index{greatest common divisor}
\index{greatest common divisor!integers}
\[
\gcd(a,b) = \max \set{d}{d\mbox{ divides both }a \mbox{ and } b}
\]
\bit
\vitem
for definition of gcd
for general entire rings,
refer to \definitionname~\ref{definition:greatest common divisor}
\eit
\vitem
least common multiple (lcm) (of $a$ and $b$)
\index{least common multiple}
\index{least common multiple!integers}
\[
\mbox{lcm}(a,b) = \min \set{m}{\mbox{both } a \mbox{ and } b \mbox{ divides }m}
\]
\vitem
$a$ and $b$ coprime, relatively prime, mutually prime $\Leftrightarrow$ $\gcd(a,b)=1$
\eit
\vfill
\myfoilhead{Are there infinite number of prime numbers?}
\bit
\item
yes!
\vitem
proof
\bit
\vitem
assume there only exist finite number of prime numbers, \eg, $p_1 < p_2 < \cdots <p_n$\
\vitem
but then, $p_1 \cdot p_2 \cdots p_n + 1$ is prime,
but which is greater than $p_n$, hence contradiction
\eit
\eit
\vfill
\myfoilhead{Integers modulo $n$}
\begin{mydefinition}{modulo}
when $n$ divides $a-b$,
$a$, said to be \define{equivalent to} $b$ \define{modulo $n$},
denoted by
$$
a \equiv b \Mod{n}
$$
read as \define{``$a$ congruent to $b$ mod $n$''}
\end{mydefinition}
\bit
\vitem
$a\equiv b\Mod{n}$ and $c\equiv d\Mod{n}$ imply
\bit
\vitem
$a+c\equiv b+d \Mod{n}$
\vitem
$ac\equiv bd \Mod{n}$
\eit
\eit
\begin{mydefinition}{congruence class}
\index{congruence class}
\index{integers!congruence class}
\index{congruence class!integers}
\index{residue class under modulo}
\index{integers!residue class under modulo}
\index{residue class under modulo!integers}
classes determined by modulo relation,
called \define{congruence} or \define{residue class under modulo}
\end{mydefinition}
\begin{mydefinition}{integers modulo n}
\index{integers modulo $n$}
\index{integers mod $n$}
\index{integers!integers modulo $n$}
\index{integers!integers mod $n$}
set of equivalence classes under modulo,
denoted by \define{$\integers/n \integers$},
called \define{integers modulo $n$} or \define{integers mod $n$}
\end{mydefinition}
\myfoilhead{Euler's theorem}
\begin{mydefinition}{Euler's totient function}
\index{Euler, Leonhard!Euler's totient function}
\index{Euler, Leonhard!phi-function}
\index{Euler phi-function}
\index{Euler's totient function}
for $n\in\naturals$,
$$
\varphi(n)
= (p_1-1)p_1^{e_1-1} \cdots (p_k-1)p_k^{e_k-1}
= n \prod_{\mathrm{prime}\ p\ \mathrm{dividing}\ n} (1-1/p)
$$
called \define{Euler's totient function},
also called \define{Euler $\varphi$-function}
\index{Euler $\varphi$-function}
\index{Euler, Leonhard!$\varphi$-function}
\end{mydefinition}
\bit
\item \eg, $\varphi(12) = \varphi(2^2\cdot 3^1) = 1\cdot2^1\cdot 2\cdot3^0 = 4$,
$\varphi(10) = \varphi(2^1\cdot5^1) = 1\cdot2^0\cdot 4\cdot 5^0 =4$
\eit
\begin{mytheorem}{Euler's theorem - number theory}
\index{Euler's theorem}
\index{Euler, Leonhard!Euler's theorem}
for coprime $n$ and $a$
$$
a^{\varphi(n)} \equiv 1 \Mod{n}
$$
\end{mytheorem}
\bit
\item
\eg, $5^4 \equiv 1 \Mod{12}$ whereas $4^4 \equiv 4 \neq 1 \Mod{12}$
\eit
\bit
\vitem
\eemph{Euler's theorem underlies RSA cryptosystem, which is pervasively used in internet communication}
\eit
\titlefoil{Trigonometric functions}{Trigonometric-functions}
\myfoilhead{Trigonometric functions}
\bit
\item
real functions relating angle of right-angled triangle to ratios of two side lengths,
called \define{trigonometric functions}
\index{trigonometric functions}
\bit
\item
also called circular functions, angle functions, or goniometric functions
\index{trigonometric functions!circular functions}
\index{trigonometric functions!angle functions}
\index{trigonometric functions!goniometric functions}
\eit
\vitem
widely used in sciences related to
geometry, such as
navigation, solid mechanics, celestial mechanics, geodesy, \etc\
\vitem
trigonometric functions most widely used in modern mathematics are
\bit
\item
sine, cosine, tangent functions - denoted by $\sin$, $\cos$, $\tan$ respectively
\item
reciprocals - cosecant, secant, cotangent - denoted by $\csc$, $\sec$, $\cot$ respectively
\eit
\eit
\myfoilhead{Definitions of trigonometric functions}
\bit
\item basic definitions (\figref{definition of trigonometric functions})\
\index{trigonometric functions!sine}
\index{trigonometric functions!cosine}
\index{trigonometric functions!tangent}
\index{trigonometric functions!cosecant}
\index{trigonometric functions!secant}
\index{trigonometric functions!cotangent}
\[
\begin{array}{lll}
\sin \theta = a/c = \mbox{opposite}/\mbox{hypotenuse},
&
\csc \theta = c/a = 1/\sin \theta
\\
\cos \theta = b/c = \mbox{adjacent}/\mbox{hypotenuse},
&
\sec \theta = c/b = 1/\cos \theta
\\
\tan \theta = a/b = \mbox{opposite}/\mbox{adjacent} = \sin \theta / \cos \theta,
&
\cot \theta = b/a = 1/\tan \theta
\end{array}
\]
\eit
\vfill
\begin{figure}
\begin{center}
\mypsfrag{a}{$a$ - opposite}
\mypsfrag{b}{$\begin{array}{c}b\\\mbox{adjacent}\end{array}$}
\mypsfrag{c}{hypotenuse - $c$}
\mypsfrag{A}{$A$}
\mypsfrag{B}{$B$}
\mypsfrag{C}{$C$}
\mypsfrag{theta}{$\theta$}
\includegraphics[width=.4\textwidth]{figures/right-angled-triangle}
\idxfig{definition of trigonometric functions}
\label{fig:definition of trigonometric functions}
\end{center}
\end{figure}
\labelfoilhead{Pythagorean theorem and trigonometric functions}
\bit
\item
Pythagorean theorem implies $a^2 + b^2 = c^2$, thus $(a/c)^2 + (b/c)^2 = 1$
(\figref{triangle for pythagorean theorem}),
hence
$$
\sin^2 \theta + \cos^2 \theta = 1
\Leftrightarrow
\tan^2 \theta + 1 = \sec^2 \theta
\Leftrightarrow
1 + \cot^2 \theta = \csc^2 \theta
$$
\eit
\vfill
\begin{figure}
\begin{center}
\mypsfrag{a}{$a$ - opposite}
\mypsfrag{b}{$\begin{array}{c}b\\\mbox{adjacent}\end{array}$}
\mypsfrag{c}{hypotenuse - $c$}
\mypsfrag{A}{$A$}
\mypsfrag{B}{$B$}
\mypsfrag{C}{$C$}
\mypsfrag{theta}{$\theta$}
\includegraphics[width=.4\textwidth]{figures/right-angled-triangle}
\idxfig{triangle for pythagorean theorem}
\label{fig:triangle for pythagorean theorem}
\end{center}
\end{figure}
\myfoilhead{Unit for angles - radians vs degrees}
\bit
\item
for trigonometric functions, most times use \emph{radians} for units of angles instead of \emph{degrees}
\vitem
why use radians? - cuz can write following (clean) formula, \eg,
\bit
\item
{length of arc having central angle} $\theta$ { (in radius)} = $\theta r$
\item
{area of sector having central angle} $\theta$ { (in radius)} = $\theta r^2 /2$
\eit
\vitem
both proportional to each other - very easy to convert
$$
d^\circ = \frac{d}{180} \pi \mbox{ radians}
\Leftrightarrow
\theta \mbox{ radians} = \left( \frac{180}{\pi} \theta \right)^\circ
$$
\vitem
\eg\ (drop ``radians'' when writing)
\bit
\item
$0 = 0^\circ$,
$\pi/12 = 15^\circ$,
$2\pi/12 = 30^\circ$,
$3\pi/12 = 45^\circ$,
$4\pi/12 = 60^\circ$,
\item
$5\pi/12 = 75^\circ$,
$6\pi/12 = 90^\circ$
$7\pi/12 = 105^\circ$,
$8\pi/12 = 120^\circ$,
\item
$9\pi/12 = 135^\circ$,
$10\pi/12 = 150^\circ$,
$11\pi/12 = 165^\circ$,
$12\pi/12 = 180^\circ$,
\eit
\eit
\myfoilhead{Some values of sine function}
\bit
\item
derive from \figref{triangles for special values of sin}
\begin{eqnarray*}
\sin 0 &=& \sin 0^\circ = 0
\\
\sin ({\pi}/{6}) &=& \sin 30^\circ = 1/2
\\
\sin ({\pi}/{4}) &=& \sin 45^\circ = 1/\sqrt{2}
\\
\sin ({\pi}/{3}) &=& \sin 60^\circ = \sqrt{3}/{2}
\\
\sin ({\pi}/{2}) &=& \sin 90^\circ = 1
\end{eqnarray*}
\eit
\begin{figure}
\begin{center}
\mypsfrag{1}{$1$}
\mypsfrag{2}{$2$}
\mypsfrag{sqrt 2}{$\sqrt{2}$}
\mypsfrag{sqrt 3}{$\sqrt{3}$}
\mypsfrag{30}{$\pi/6 = 30^\circ$}
\mypsfrag{45}{$\pi/4 = 45^\circ$}
\mypsfrag{60}{$\pi/3 = 60^\circ$}
\includegraphics[width=.9\textwidth]{figures/right-angled-triangle-values}
\idxfig{triangles for special values of sin}
\label{fig:triangles for special values of sin}
\end{center}
\end{figure}
\myfoilhead{Some values of cosine function}
\bit
\item
derive from \figref{triangles for special values of cos}
\begin{eqnarray*}
\cos 0 &=& \cos 0^\circ = 1
\\
\cos ({\pi}/{6}) &=& \cos 30^\circ = \sqrt{3}/2
\\
\cos ({\pi}/{4}) &=& \cos 45^\circ = 1/\sqrt{2}
\\
\cos ({\pi}/{3}) &=& \cos 60^\circ = {1}/{2}
\\
\cos ({\pi}/{2}) &=& \cos 90^\circ = 0
\end{eqnarray*}
\eit
\begin{figure}
\begin{center}
\mypsfrag{1}{$1$}
\mypsfrag{2}{$2$}
\mypsfrag{sqrt 2}{$\sqrt{2}$}
\mypsfrag{sqrt 3}{$\sqrt{3}$}
\mypsfrag{30}{$\pi/6 = 30^\circ$}
\mypsfrag{45}{$\pi/4 = 45^\circ$}
\mypsfrag{60}{$\pi/3 = 60^\circ$}
\includegraphics[width=.9\textwidth]{figures/right-angled-triangle-values}
\idxfig{triangles for special values of cos}
\label{fig:triangles for special values of cos}
\end{center}
\end{figure}
\myfoilhead{More values of trigonometric functions}
\bit
\item
below are values of trigonometric functions
for some angles
\[
\begin{array}{|c|c|c|c|c|}
\hline
\mbox{degrees} & \mbox{radians} & \sin\theta & \cos\theta & \tan\theta
\\
\hline
0 & 0^\circ & 0 & 1 & 0
\\
\pi/12 & 15^\circ & (\sqrt{6} - \sqrt{2})/4 & (\sqrt{6}+\sqrt{2})/4 & 2-\sqrt{3}
\\
\pi/6 & 30^\circ & 1/2 & \sqrt{3}/2 & 1/\sqrt{3}
\\
\pi/4 & 45^\circ & 1/\sqrt{2} & 1/\sqrt{2} & 1
\\
\pi/3 & 60^\circ & \sqrt{3}/2 & 1/2 & \sqrt{3}
\\
5\pi/12 & 75^\circ & (\sqrt{6} + \sqrt{2})/4 & (\sqrt{6}-\sqrt{2})/4 & 2+\sqrt{3}
\\
\pi/2 & 90^\circ & 1 & 0 & \pm \infty\mbox{ (cannot be defined)
}
\\
\hline
\end{array}
\]
\vitem
who can we get values for $\pi/12$ and $5\pi/12$?
\bit
\item
do not worry; we have not learned how to get it yet - but we will!
\bit
\item [-] I don't know when, though - depends on Kim :)
\eit
\eit
\eit
\vfill
\myfoilhead{Trigonometric functions as coordiates of points on unit circle}
\bit
\item
consider point on unit circle whose angle with (positive direction) of $x$-axis is $\theta$\
(\figref{trigonometric function values as Cartesian coordinates of points on unit circle})
\bit
\item
$x$-coordinate of the point is $\cos \theta$
\item
$y$-coordinate of the point is $\sin \theta$
\eit
\vitem
following this rule,
enables us to calculate trigonometric function values for every angle;
even those greater than $\pi/2=90^\circ$ or even negative angles
\vitem
in next slides, will derive several conversion formula using this
\eit
\begin{figure}
\begin{center}
\mypsfrag{xy}{$(\cos \theta, \sin\theta)$}
\mypsfrag{theta}{$\theta$}
\mypsfrag{cos theta}{$\cos\theta$}
\mypsfrag{sin theta}{$\sin\theta$}
\mypsfrag{01}{$(0,1)$}
\mypsfrag{0-1}{$(0,-1)$}
\mypsfrag{10}{$(1,0)$}
\mypsfrag{-10}{$(-1,0)$}
\includegraphics[width=.40\textwidth]{figures/py-tri-basic-psfragable}
\idxfig{trigonometric function values as Cartesian coordinates of points on unit circle}
\label{fig:trigonometric function values as Cartesian coordinates of points on unit circle}
\end{center}
\end{figure}
\myfoilhead{Conversion formula for $\cos(\pi-\theta)$, $\sin(\pi-\theta)$, $\tan(\pi-\theta)$}
\bit
\item
derive following formula from \figref{trigonometric function conversion formula for pi minus thera}
\bit
\item
$\cos(\pi-\theta) = - \cos\theta$
\item
$\sin(\pi-\theta) = \sin\theta$
\item
$\tan(\pi-\theta) = \sin(\pi-\theta)/\cos(\pi-\theta) = -\tan \theta$
\eit
\eit
\begin{figure}
\begin{center}
\mypsfrag{xy}{$(\cos \theta, \sin\theta)$}
\mypsfrag{xy1}{$(\cos (\pi-\theta), \sin(\pi-\theta))$}
\mypsfrag{theta}{$\theta$}
\mypsfrag{cos theta}{$\cos\theta$}
\mypsfrag{cos theta0}{$(\cos\theta,0)$}
\mypsfrag{-cos theta0}{$(-\cos\theta,0)$}
\mypsfrag{0sin theta}{$(0, \sin\theta)$}
\mypsfrag{sin theta}{$\sin\theta$}
\mypsfrag{pi-theta}{$\pi-\theta$}
\mypsfrag{01}{$(0,1)$}
\mypsfrag{0-1}{$(0,-1)$}
\mypsfrag{10}{$(1,0)$}
\mypsfrag{-10}{$(-1,0)$}
\includegraphics[width=0.65\textwidth]{figures/py-pi-minus-theta-psfragable}
\idxfig{trigonometric function conversion formula for $\pi - \theta$}
\label{fig:trigonometric function conversion formula for pi minus thera}
\end{center}
\end{figure}
\myfoilhead{Conversion formula for $\cos(-\theta)$, $\sin(-\theta)$, $\tan(-\theta)$}
\bit
\item
derive following formula from \figref{trigonometric function conversion formula for minus theta}\
\bit
\item
$\cos(-\theta) = \cos\theta$
\item
$\sin(-\theta) = -\sin\theta$
\item
$\tan(-\theta) = \sin(-\theta)/\cos(-\theta) = -\tan \theta$
\eit
\eit
\begin{figure}
\begin{center}
\mypsfrag{xy}{$(\cos \theta, \sin\theta)$}
\mypsfrag{xy1}{$(\cos (-\theta), \sin(-\theta))$}
\mypsfrag{theta}{$\theta$}
\mypsfrag{-theta}{$-\theta$}
\mypsfrag{cos theta}{$\cos\theta$}
\mypsfrag{sin theta}{$\sin\theta$}
\mypsfrag{pi-theta}{$\pi-\theta$}
\mypsfrag{01}{$(0,1)$}
\mypsfrag{0-1}{$(0,-1)$}
\mypsfrag{10}{$(1,0)$}
\mypsfrag{-10}{$(-1,0)$}
\includegraphics[width=0.4\textwidth]{figures/py-minus-theta-psfragable}
\idxfig{trigonometric function conversion formula for $- \theta$}
\label{fig:trigonometric function conversion formula for minus theta}
\end{center}
\end{figure}
\myfoilhead{Conversion formula for $\cos(\theta+\pi/2)$, $\sin(\theta+\pi/2)$, $\tan(\theta+\pi/2)$}
\bit
\item
derive following formula from \figref{tri conversion rules for theta plus pi over 2}
\bit
\item
$\cos(\theta+\pi/2) = -\sin\theta$
\item
$\sin(\theta+\pi/2) = \cos\theta$
\item
$\tan(\theta+\pi/2) = \sin(\theta+\pi/2)/\cos(\theta+\pi/2) = -\cos\theta/\sin\theta= -\cot \theta$
\eit
\eit
\begin{figure}
\begin{center}
\mypsfrag{xy}{$(\cos \theta, \sin\theta)$}
\mypsfrag{xy1}{$(\cos(\theta+\pi/2),\sin(\theta+\pi/2))$}
\mypsfrag{theta}{$\theta$}
\mypsfrag{theta2}{$\theta+\pi/2$}
\mypsfrag{0cos theta}{$(0, \cos \theta)$}
\mypsfrag{-sin theta0}{$(-\sin \theta,0)$}
\mypsfrag{0sin theta}{$(0, \sin \theta)$}
\mypsfrag{cos theta0}{$(\cos \theta,0)$}
\includegraphics[width=0.55\textwidth]{figures/py-theta-plus-half-pi-psfragable}
\idxfig{trigonometric function conversion formula for $\theta+\pi/2$}
\label{fig:tri conversion rules for theta plus pi over 2}
\end{center}
\end{figure}
\iffalse
\myfoilhead{Conversion formula for trigonometric functions}
\bit
\item
can consider many other cases from similar drawings
\vitem
below table shows some formula
\[
\begin{array}{|c|r|r|r|}
\hline
\xi & \sin\xi & \cos\xi & \tan\xi
\\
\hline
\theta +\pi/2& \cos\theta & -\sin\theta & -\cot\theta
\\
\theta + \pi & -\sin\theta & -\cos\theta & \tan\theta
\\
\theta + 3\pi/2 & -\cos\theta & \sin\theta & -\cot\theta
\\
\theta + 2\pi & \sin\theta & \cos\theta & \tan\theta
\\
-\theta & -\sin\theta & \cos\theta & -\tan\theta
\\
\pi/2-\theta & \cos\theta & \sin\theta & \cot\theta
\\
\pi-\theta & \sin\theta & -\cos\theta & -\tan\theta
\\
3\pi/2-\theta & -\cos\theta & -\sin\theta & \cot\theta
\\
2\pi-\theta & -\sin\theta & \cos\theta & -\tan\theta
\\
\hline
\end{array}
\]
\vitem
\eemph{actually you can derive every formula using only these three:
$\sin(-\theta)=-\sin(\theta)$, $\sin(\theta + \pi/2) = \cos(\theta)$, $\sin(\theta+\pi)=-\sin(\theta)$; try to figure out how!}
\eit
\myfoilhead{Evaluation using conversion formula - 1}
\bit
\item
let's try to calculate trigonometric function values for various angles
given four values below only - can we?
\[
\sin 0 = 0,
\ \
\sin(\pi/6) = 1/2,
\ \
\sin(\pi/4) = 1/\sqrt{2},
\ \
\sin(\pi/3) = \sqrt{3}/2,
\]
\vitem
examples
\[
\begin{array}{cccl}
\cos (30^\circ) &=& \cos (\pi/6) &= \cos(\pi/2 - \pi/3) = \sin(\pi/3) = \sqrt{3}/2
\\
\tan (30^\circ) &=& \tan (\pi/6) &= \sin(\pi/6) / \cos(\pi/6) = 1/ \sqrt{3}
\\
\\
\cos (60^\circ) &=& \cos (\pi/3) &= \cos(\pi/2 - \pi/6) = \sin(\pi/6) = {1}/2
\\
\tan (60^\circ) &=& \tan (\pi/3) &= \sin(\pi/3) / \cos(\pi/3) = \sqrt{3}
\\
\\
\sin (120^\circ) &=& \sin(2\pi/3) &= \sin(\pi - \pi/3) = \sin(\pi/3) = \sqrt{3}/2
\\
\cos (120^\circ) &=& \cos(2\pi/3) &= \cos(\pi - \pi/3) = -\cos(\pi/3) = -1/2
\\
\tan (120^\circ) &=& \tan (2\pi/3) &= \sin(2\pi/3) / \cos(2\pi/3) = -\sqrt{3}
\end{array}
\]
\eit
\myfoilhead{Evaluation using conversion formula - 2}
\bit
\item
more examples
\[
\begin{array}{cccl}
\sin(-45^\circ ) &=&\sin(-\pi/4) &= - \sin(\pi/4) = - 1/\sqrt{2}
\\
\cos(-45^\circ ) &=&\cos(-\pi/4) &= \cos(\pi/4) = 1/\sqrt{2}
\\
\tan(-45^\circ) &=& \tan(-\pi/4) &= \sin(-\pi/4) / \cos(-\pi/4) = -1
\\
\\
\sin(210^\circ ) &=&\sin(7\pi/6) &= \sin(\pi/6 + \pi) = -\sin(\pi/6) = - 1/{2}
\\
\cos(210^\circ ) &=&\cos(7\pi/6) &= \cos(\pi/6 + \pi) = -\cos(\pi/6) = - \sqrt{3}/{2}
\\
\tan(210^\circ) &=& \tan(7\pi/6) &= \sin(7\pi/6) / \cos(7\pi/6) = 1/\sqrt{3}
\\
\\
\sin(315^\circ ) &=&\sin(7\pi/4) &= \sin(2\pi - \pi/4) = -\sin(\pi/4) = - 1/\sqrt{2}
\\
\cos(315^\circ ) &=&\cos(7\pi/4) &= \cos(2\pi - \pi/4) = \cos(\pi/4) = 1/\sqrt{2}
\\
\tan(315^\circ) &=& \tan(7\pi/4) &= \sin(7\pi/4) / \cos(7\pi/4) = -1
\end{array}
\]
\eit
\myfoilhead{Figuring out conversion formula using graphs without memorizing them}
\bit
\item
\bit
\item
\eit
\vitem
\bit
\item
\eit
\eit
\myfoilhead{Law of sines}
\begin{mylaw}{law of sines}
$$
\frac{a}{\sin A}
=
\frac{b}{\sin B}
=
\frac{c}{\sin C}
$$
\end{mylaw}
\vfill
\begin{figure}
\begin{center}
\mypsfrag{A}{$A$}
\mypsfrag{B}{$B$}
\mypsfrag{C}{$C$}
\mypsfrag{a}{$a$}
\mypsfrag{b}{$b$}
\mypsfrag{c}{$c$}
\includegraphics[width=.30\textwidth]{figures/py-triangle-psfragable}
\idxfig{triangle}
\label{fig:triangle}
\end{center}
\end{figure}
\vfill
\begin{proof}
From \figref{triangle}, the height of triangle when $a$ is chosen as base of the triangle is $c\sin B = b\sin C$,
hence $b/\sin B = c/\sin C$.
Applying this to another side gives us $a/\sin A = b/\sin B = c/\sin C$.
\eemph{Please make sure yourself whether this is true for obtuse triangles, too!}
\end{proof}
\myfoilhead{Law of cosines}
\begin{mylaw}{law of cosines}
\begin{eqnarray*}
a^2 &=& b^2 + c^2 - 2bc \cos A
\\
b^2 &=& c^2 + a^2 - 2ca \cos B
\\
c^2 &=& a^2 + b^2 - 2ab \cos C
\end{eqnarray*}
\end{mylaw}
\vfill
\begin{proof}
From \figref{triangle}, applying Pythagorean theorem to side, $b$,
yields
\begin{eqnarray*}
b^2
& = &
(c\sin B)^2 + (a-c\cos B)^2
\\
&=&
c^2 (\sin^2 B + \cos^2 B) - 2ac \cos B + a^2
=
a^2 + c^2 - 2ac \cos B
\end{eqnarray*}
where $\sin^2 B + \cos^2 B=1$ shown \foilref{Pythagorean theorem and trigonometric functions}\ is used.
Similar equations arise by choosing the side of length $a$ or the side of length $c$
to apply Pythagorean theorem.
\eemph{Please make sure yourself whether this is true for obtuse triangles, too!}
\end{proof}
\myfoilhead{Quiz for Beth - triangulation}
\bit
\item
suppose you want to know the distance between the target point, $X$, and line, $\overline{AB}$ (\figref{triangulation}).
however you are on the other side of the river,
which is so wide that you cannot cross.
but you can measure two angles $\alpha$ and $\beta$ and distance, $d$, between $A$ and $B$.
\vitem
do you think you can calculate (exactly) distance between $X$ and $\overline{AB}$,
\ie, length of dotted line in \figref{triangulation}?
if so, how?
(hint: would ``law of sines'' be helpful?)
\eit
\begin{figure}
\begin{center}
\mypsfrag{a}{$d$}
\mypsfrag{b}{}
\mypsfrag{c}{}
\mypsfrag{A}{$X$ (target)}
\mypsfrag{B}{$A$}
\mypsfrag{C}{$B$}
\mypsfrag{alpha}{$\alpha$}
\mypsfrag{beta}{$\beta$}
\includegraphics[width=.5\textwidth]{figures/py-triangulation-psfragable}
\idxfig{triangulation}
\label{fig:triangulation}
\end{center}
\end{figure}
\myfoilhead{Trigonometric identities}
\begin{myformula}{trigonometric identities}
\begin{eqnarray*}
\sin(\alpha + \beta) &=& \sin(\alpha)\cos(\beta) + \cos(\alpha)\sin(\beta)
\\
\cos(\alpha + \beta) &=& \cos(\alpha)\cos(\beta) - \sin(\alpha)\sin(\beta)
\\
\tan(\alpha + \beta) &=& (\tan\alpha + \tan \beta) / (1-\tan\alpha\tan\beta)
\\
\sin 2\theta
&=&
2\sin\theta\cos\theta
\\
\cos 2\theta
&=&
\cos^2\theta- \sin^2\theta = 2\cos^2\theta - 1 = 1-2\sin^2\theta
\\
\tan2\theta &=& 2\tan\theta / (1-\tan^2\theta)
\\
\cos \alpha + \cos \beta
&=&
2\cos \left(\frac{\alpha + \beta}{2}\right) \cos \left(\frac{\alpha - \beta}{2}\right)
\\
\sin \alpha + \sin \beta
&=&
2\sin \left(\frac{\alpha + \beta}{2}\right) \cos \left(\frac{\alpha - \beta}{2}\right)
\end{eqnarray*}
\end{myformula}
-- note all other formula can be drived from first one
\fi
\ifthenelse{\equal{\value{numsectionsforwhichproofexists}}{0}}{}{
}
\TITLEFOIL{References}{References}
\myfoilhead{}
\bibliographystyle{alpha}
\bibliography{../../FrequentlyUsed/latex/mybib}
\end{document}