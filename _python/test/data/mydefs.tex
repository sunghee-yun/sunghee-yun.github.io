%% users inputting this file should define below yes/no commands

%\def\pdfpsfrag{yes or no}
%\def\reusepsfragpdf{yes or no}
%\def\foiltex{yes or no}
%\def\algfontmode{yes or no}

%% yes/no conditional 

\usepackage{ifthen}
\newcommand\yesnoexec[3]{
\ifthenelse{\equal{#1}{yes}}
{#2%
}{
	\ifthenelse{\equal{#1}{no}}{#3}{\errmessage{{#1} should be either "yes" or "no"}}
}
}

\ifdefined\pdfpsfrag \else
\def\pdfpsfrag{no}
\fi

\yesnoexec{\pdfpsfrag}{%
\ifdefined\reusepsfragpdf \else
\def\reusepsfragpdf{no}
\fi
}{}

\ifdefined\foiltex \else
\def\foiltex{no}
\fi

\ifdefined\algfontmode \else
\def\algfontmode{no}
\fi

\yesnoexec{\algfontmode}{%
%\usepackage[subscriptcorrection,slantedGreek,nofontinfo,mtpcal,mtphrb,mtpccal,mtpscr,mtpfrak]{mtpro2}
\usepackage[mtpccal,mtpscr]{mtpro2}
%\usepackage[mtpccal,mtpfrak,mtpscr]{mtpro2}
%\usepackage[mathscr]{euscript} % mathscr
\usepackage{amsfonts} % mathfrak

\newcommand\mathalgfont[1]{\mathcal{#1}}
\newcommand\mathcalfont[1]{\mathscr{#1}}
}{%
\usepackage{amssymb} % mathfrak
\usepackage{mathrsfs} % mathscr

\newcommand\mathalgfont[1]{\mathscr{#1}}
\newcommand\mathcalfont[1]{\mathcal{#1}}
}


%% PICTURE ENVIRONMENT

\newcommand\rect[6]{%
\put(#1,#2){\line(1,0){#5}}%
\put(#1,#2){\line(0,1){#6}}%
\put(#1,#4){\line(1,0){#5}}%
\put(#3,#2){\line(0,1){#6}}%
}

\newcommand\crect[7]{%
\put(#1,#2){\color{#7}\line(1,0){#5}}%
\put(#1,#2){\color{#7}\line(0,1){#6}}%
\put(#1,#4){\color{#7}\line(1,0){#5}}%
\put(#3,#2){\color{#7}\line(0,1){#6}}%
}

\newcommand\pptext[3]{
	\put(#1){\makebox(0,0)[#2]{#3}}
}

\newcommand\ptext[4]{\pptext{#1}{#2}{\textcolor{#4}{#3}}}


%% COLOR DEFINITIONS

%\usepackage[]{hyperref} % i put below due to conflict between hyperref package and \newtheorem\theorem
\usepackage[
  colorlinks=true,
  linkcolor=black,
  urlcolor=blue,
  citecolor=blue,
%  pdftitle={Your Document Title},
  pdfauthor={Sunghee Yun}
]{hyperref}

\usepackage[dvipsnames]{xcolor} % for below color definitions

\definecolor{rgb-60-60-60}{rgb}{.6, .6, .6}
\definecolor{rgb-10-80-10}{rgb}{.1, .8, .1}
\definecolor{crimson}{rgb}{0.86, 0.08, 0.24}
\definecolor{dimgray}{rgb}{0.41, 0.41, 0.41}
\definecolor{lavendergray}{rgb}{0.77, 0.76, 0.82}
\definecolor{cobalt}{rgb}{0.0, 0.28, 0.67}
\definecolor{unitednationsblue}{rgb}{0.36, 0.57, 0.9}
\definecolor{ultramarine}{rgb}{0.07, 0.04, 0.56}
\definecolor{ultramarineblue}{rgb}{0.25, 0.4, 0.96}
\definecolor{ufogreen}{rgb}{0.24, 0.82, 0.44}
\definecolor{tropicalrainforest}{rgb}{0.0, 0.46, 0.37}
\definecolor{tiffanyblue}{rgb}{0.04, 0.73, 0.71}
\definecolor{springgreen}{rgb}{0.0, 1.0, 0.5}
\definecolor{Lime-Green}{HTML}{32cd32}
\definecolor{saddle-brown}{RGB}{139,69,19}
\definecolor{sienna}{RGB}{160,82,45}
\definecolor{chocolate}{RGB}{210,105,30}

% SOME COMMANDS & ENVIRONMENTS FOR MAKING LECTURE NOTES

\newcounter{oursection}
\newcommand{\oursection}[1]{
 \addtocounter{oursection}{1}
 \setcounter{equation}{0}
 \clearpage \begin{center} {\Huge\bfseries #1} \end{center}
 {\vspace*{0.15cm} \hrule height.3mm} \bigskip
 \addcontentsline{toc}{section}{#1}
}
\newcommand{\oursectionf}[1]{  % for use with foiltex
 \addtocounter{oursection}{1}
 \setcounter{equation}{0}
 \foilhead[-.5cm]{#1 \vspace*{0.8cm} \hrule height.3mm }
 \LogoOn
}
\newcommand{\oursectionfl}[1]{  % for use with foiltex landscape
 \addtocounter{oursection}{1}
 \setcounter{equation}{0}
 \foilhead[-1.0cm]{#1}
 \LogoOn
}

\newcounter{lecture}
\newcommand{\lecture}[1]{
 \addtocounter{lecture}{1}
 \setcounter{equation}{0}
 \setcounter{page}{1}
 \renewcommand{\theequation}{\arabic{equation}}
 \renewcommand{\thepage}{\arabic{lecture} -- \arabic{page}}
 \raggedright \sffamily \LARGE
 \cleardoublepage\begin{center}
 {\Huge\bfseries Lecture \arabic{lecture} \bigskip \\ #1}\end{center}
 {\vspace*{0.15cm} \hrule height.3mm} \bigskip
 \addcontentsline{toc}{chapter}{\protect\numberline{\arabic{lecture}}{#1}}
 \pagestyle{myheadings}
 \markboth{Lecture \arabic{lecture}}{#1}
}
\newcommand{\lecturef}[1]{
 \addtocounter{lecture}{1}
 \setcounter{equation}{0}
 \setcounter{page}{1}
 \renewcommand{\theequation}{\arabic{equation}}
 \renewcommand{\thepage}{\arabic{lecture}--\arabic{page}}
 \parindent 0pt
 \MyLogo{#1}
 \rightfooter{\thepage}
 \leftheader{}
 \rightheader{}
 \LogoOff
 \begin{center}
 {\large\bfseries Lecture \arabic{lecture} \bigskip \\ #1}
 \end{center}
 {\vspace*{0.8cm} \hrule height.3mm}
 \bigskip
}
\newcommand{\lecturefl}[1]{   % use with foiltex landscape
 \addtocounter{lecture}{1}
 \setcounter{equation}{0}
 \setcounter{page}{1}
 \renewcommand{\theequation}{\arabic{equation}}
 \renewcommand{\thepage}{\arabic{lecture}--\arabic{page}}
 \addtolength{\topmargin}{-1.5cm}
 \raggedright
 \parindent 0pt
 \rightfooter{\thepage}
 \leftheader{}
 \rightheader{}
 \LogoOff
 \begin{center}
 {\Large \bfseries Lecture \arabic{lecture} \\*[\bigskipamount] {#1}}
 \end{center}
 \MyLogo{#1}
}


% NORMAL SENTENCES

% acronyms

\newcommand{\eg}{{\it e.g.}}
\newcommand{\ie}{{\it i.e.}}
\newcommand{\etc}{{\it etc.}}
\newcommand{\cf}{{\it cf.}}

\newcommand{\graystrikethrough}[1]{\textcolor{rgb-60-60-60}{\sout{#1}}}

\newcommand\blfootnote[1]{%
	\begingroup
	\renewcommand\thefootnote{}\footnote{#1}%
	\addtocounter{footnote}{-1}%
	\endgroup
}

\newcommand\brfootnote[1]{\let\thefootnote\relax\footnotetext}

% THEOREMS AND SUCH

\usepackage{thmtools}

\yesnoexec{\foiltex}{%
	\declaretheorem{axiom}
	\declaretheorem{law}
	\declaretheorem{principle}
	\declaretheorem{definition}
	\declaretheorem{theorem}
	\declaretheorem{lemma}
	\declaretheorem{proposition}
	\declaretheorem{corollary}
	\declaretheorem{conjecture}
	\declaretheorem{inequality}
	\declaretheorem{formula}
	\declaretheorem{algorithm}
}{%
	\declaretheorem[numberwithin=section]{axiom}
	\declaretheorem[numberwithin=section]{law}
	\declaretheorem[numberwithin=section]{principle}
	\declaretheorem[numberwithin=section]{definition}
	\declaretheorem[numberwithin=section]{theorem}
	\declaretheorem[numberwithin=section]{lemma}
	\declaretheorem[numberwithin=section]{proposition}
	\declaretheorem[numberwithin=section]{corollary}
	\declaretheorem[numberwithin=section]{conjecture}
	\declaretheorem[numberwithin=section]{inequality}
	\declaretheorem[numberwithin=section]{formula}
	\declaretheorem[numberwithin=section]{algorithm}
}

\newcommand\axiomname{Axiom}
\newcommand\lawname{Law}
\newcommand\principlename{Principle}
\newcommand\definitionname{Definition}
\newcommand\theoremname{Theorem}
\newcommand\lemmaname{Lemma}
\newcommand\propositionname{Proposition}
\newcommand\corollaryname{Corollary}
\newcommand\conjecturename{Conjecture}
\newcommand\inequalityname{Ineq}
\newcommand\formulaname{Formula}
\newcommand\algorithmname{Algorithm}

% alias
\newcommand\definename{\definitionname}

\newenvironment{proof}{\begin{quote}\textit{Proof}:}{\end{quote}}
\newenvironment{solution}{\begin{quote}\textit{Solution}:}{\end{quote}}
\newenvironment{pcode}{\begin{quote}\textit{Python source code}:}{\end{quote}}
\newenvironment{names}{\begin{itshape}}{\end{itshape}}

\newcommand{\qed}{{\bf Q.E.D.}}
\renewcommand{\qed}{\rule[-.5ex]{.5em}{2ex}}
\newcommand{\textfn}[1]{\textsl{#1}}

% table

\newcommand{\tparbox}[2]{%
{\parbox[c]{#1}{\center\vspace{-.4\baselineskip}{#2}\vspace{.3\baselineskip}}}}

% software

\newenvironment{code}{\begin{quote}\begin{tt}}{\end{tt}\end{quote}}

% math

\newcommand{\pluseq}{\mathrel{+}=}

\newcommand{\bmyeq}{\[}
\newcommand{\emyeq}{\]}
\newcommand{\bmyeql}[1]{\begin{equation}\label{#1}}
\newcommand{\emyeql}{\end{equation}}
%\newenvironment{myeq}{\[}{\]}
%\newenvironment{myeql}[1]{\begin{equation}\label{#1}}{\end{equation}}

\newcommand{\onehalf}{\ensuremath{\frac{1}{2}}}
\newcommand{\onethird}{\ensuremath{\frac{1}{3}}}
\newcommand{\onefourth}{\frac{1}{4}}
\newcommand{\sumft}[2]{\sum_{#1}^{#2}}
\newcommand{\sumioneton}{\sumionetok{n}}
\newcommand{\sumionetok}[1]{\sum_{i=1}^#1}
\newcommand{\sumoneto}[2]{\sum_{#1=1}^{#2}}
\newcommand{\sumoneton}[1]{\sumoneto{#1}{n}}
\newcommand{\prodoneto}[2]{\prod_{#1=1}^{#2}}
\newcommand{\prodoneton}[1]{\prodoneto{#1}{n}}

\newcommand{\listoneto}[1]{\ensuremath{1,2,\ldots,#1}}
\newcommand{\diagxoneto}[2]{\ensuremath{\diag({#1}_1,{#1}_2,\ldots,{#1}_{#2})}}
\newcommand{\setxoneto}[2]{\ensuremath{\{\listxoneto{#1}{#2}\}}}
\newcommand{\listxoneto}[2]{\ensuremath{{#1}_1,{#1}_2,\ldots,{#1}_{#2}}}
\newcommand{\setoneto}[1]{\ensuremath{\{1,2,\ldots,#1\}}}

\newcommand{\diagmat}[2]{\diagoneto{#1}{#2}}

\newcommand{\setoneton}[1]{\setoneton{#1}}
\newcommand{\setxtok}[2]{\setxoneto{#1}{#2}}

% matrices

\newcommand{\colvectwo}[2]{\ensuremath{\begin{my-matrix}{c}{#1}\\{#2}\end{my-matrix}}}
\newcommand{\colvecthree}[3]{\ensuremath{\begin{my-matrix}{c}{#1}\\{#2}\\{#3}\end{my-matrix}}}
\newcommand{\colvecfour}[4]{\ensuremath{\begin{my-matrix}{c}{#1}\\{#2}\\{#3}\\{#4}\end{my-matrix}}}
\newcommand{\rowvectwo}[2]{\ensuremath{\begin{my-matrix}{cc}{#1}&{#2}\end{my-matrix}}}
\newcommand{\rowvecthree}[3]{\ensuremath{\begin{my-matrix}{ccc}{#1}&{#2}&{#3}\end{my-matrix}}}
\newcommand{\rowvecfour}[4]{\ensuremath{\begin{my-matrix}{cccc}{#1}&{#2}&{#3}&{#4}\end{my-matrix}}}
\newcommand{\diagtwo}[2]{\ensuremath{\begin{my-matrix}{cc}{#1}&0\\0& {#2}\end{my-matrix}}}
\newcommand{\mattwotwo}[4]{\ensuremath{\begin{my-matrix}{cc}{#1}&{#2}\\{#3}&{#4}\end{my-matrix}}}
\newcommand{\bigmat}[9]{\ensuremath{\begin{my-matrix}{cccc} #1&#2&\cdots&#3\\ #4&#5&\cdots&#6\\ \vdots&\vdots&\ddots&\vdots\\ #7&#8&\cdots&#9 \end{my-matrix}}}
%\newcommand{\matdotff}[9]{\matff{#1}{#2}{\cdots}{#3}{#4}{#5}{\cdots}{#6}{\vdots}{\vdots}{\ddots}{\vdots}{#7}{#8}{\cdots}{#9}}

\newcommand{\matthreethree}[9]{%
	\begin{my-matrix}{ccc}%
	{#1}&{#2}&{#3}%
	\\{#4}&{#5}&{#6}%
	\\{#7}&{#8}&{#9}%
	\end{my-matrix}%
}
\newcommand{\matthreethreeT}[9]{%
	\matthreethree%
	{#1}{#4}{#7}%
	{#2}{#5}{#8}%
	{#3}{#6}{#9}%
}

\newenvironment{my-matrix}{\left[\begin{array}}{\end{array}\right]}

\newcommand{\mbyn}[2]{\ensuremath{#1\times #2}}
\newcommand{\realmat}[2]{\ensuremath{\reals^{\mbyn{#1}{#2}}}}
\newcommand{\realsqmat}[1]{\ensuremath{\reals^{\mbyn{#1}{#1}}}}
\newcommand{\defequal}{\triangleq}
\newcommand\symset[1]{\ensuremath{\mbox{\bf S}^{#1}}}
\newcommand\possemidefset[1]{\ensuremath{\mbox{\bf S}_+^{#1}}}
\newcommand\posdefset[1]{\ensuremath{\mbox{\bf S}_{++}^{#1}}}

% spaces for integral, etc.

\newcommand{\dspace}{\,}
\newcommand{\dx}{{\dspace dx}}
\newcommand{\dy}{{\dspace dy}}
\newcommand{\dt}{{\dspace dt}}
\newcommand{\intspace}{\!\!}
\newcommand{\sqrtspace}{\,}
\newcommand{\aftersqrtspace}{\sqrtspace}
\newcommand{\dividespace}{\!}


\usepackage{amsmath}

%\DeclareMathOperator\support{support}
\newcommand\support{\operatorname*{\bf support}}
%\newcommand\support{{\mathop{\bf support}}}


\DeclareMathOperator\Img{Im}
\DeclareMathOperator\Ker{Ker}
\DeclareMathOperator\Gal{Gal}
\DeclareMathOperator\Map{Map}
\DeclareMathOperator\Aut{Aut}
\DeclareMathOperator\End{End}
\DeclareMathOperator\Irr{Irr}
\DeclareMathOperator\ev{ev}
\DeclareMathOperator\affinehull{\bf aff}
\DeclareMathOperator\relint{\bf relint}
\DeclareMathOperator\cvxhull{\bf Conv}
\DeclareMathOperator\boundary{\bf bd}
\DeclareMathOperator\epi{\bf epi}
\DeclareMathOperator\hypo{\bf hypo}

\newcommand{\arginf}{\mathop{\mathrm{arginf}}}
\newcommand{\argsup}{\mathop{\mathrm {argsup}}}
\newcommand{\argmin}{\mathop{\mathrm {argmin}}}
\newcommand{\argmax}{\mathop{\mathrm {argmax}}}

\newcommand\ereals{\reals\cup\{-\infty,\infty\}}
\newcommand{\reals}{\ensuremath{\mbox{\bf R}}}
\newcommand{\preals}{\ensuremath{\reals_{+}}}
\newcommand{\prealk}[1]{\ensuremath{\reals_{+}^{#1}}}
\newcommand{\ppreals}{\ensuremath{\reals_{++}}}
\newcommand{\pprealk}[1]{\ensuremath{\reals_{++}^{#1}}}
\newcommand{\complexes}{\ensuremath{\mbox{\bf C}}}
\newcommand{\integers}{{\mbox{\bf Z}}}
\newcommand{\naturals}{{\mbox{\bf N}}}
\newcommand{\rationals}{{\ensuremath{\mbox{\bf Q}}}}

\newcommand{\realspace}[2]{\reals^{#1\times #2}}
\newcommand{\compspace}[2]{\complexes^{#1\times #2}}

\newcommand{\identity}{\mbox{\bf I}}
\newcommand{\nullspace}{{\mathcalfont N}}
\newcommand{\range}{{\mathcalfont R}}

% vectors, sets, etc.

\newcommand{\one}{\mathbf{1}}
\newcommand{\ones}{\mathbf 1}
\newcommand{\ordinal}{^{\mathrm{th}}}
\newcommand{\set}[2]{\ensuremath{\{#1|#2\}}}
\newcommand{\bigset}[2]{\ensuremath{\left\{{#1}\left|{#2}\right.\right\}}}
\newcommand{\bigsetl}[2]{\left\{\left.{#1}\right|{#2}\right\}}

% operators

\newcommand{\Expect}{\mathop{\bf E{}}}
\newcommand{\Var}{\mathop{\bf  Var{}}}
\newcommand{\Cov}{\mathop{\bf Cov}}
\newcommand{\Prob}{\mathop{\bf Prob}}
%\newcommand\Prob{\operatorname*{\bf P}}
\newcommand\prob[1]{\Prob\left\{#1\right\}}
\renewcommand\prob[1]{\Prob\left(#1\right)}

\newcommand{\smallo}{{\mathop{\bf o}}}

\newcommand{\jac}{{\mathcalfont{J}}}
\newcommand{\diag}{\mathop{\bf diag}}
\newcommand{\Rank}{\mathop{\bf Rank}}
\newcommand{\rank}{\mathop{\bf rank}}
\newcommand{\dimn}{\mathop{\bf dim}}
\newcommand{\Tr}{\mathop{\bf Tr}} % trace
\newcommand{\dom}{\mathop{\bf dom}}
\newcommand{\Det}{\det}
\newcommand{\adj}{\mathop{\bf adj}}
\newcommand{\minor}{\mathop{\bf minor}}
%\newcommand{\Det}{{\mathop{\bf Det}}}
%\newcommand{\determinant}[1]{|#1|}
\newcommand{\sign}{{\mathop{\bf sign}}}
\newcommand{\dist}{{\mathop{\bf dist}}}

% probability space

\newcommand{\probsubset}{{\mathcalfont{P}}}
\newcommand{\eset}{{\mathcalfont{E}}}

\newcommand{\probspace}{{\Omega}}

% optimization

% phrases

\newcommand\iaoi{\emph{if and only if}}
\newcommand\wrt{with respect to}

% mathematicians' names

\newcommand\cara{Carath\'{e}odory}

% FOR ANALYSIS
% names of families, collections, sets, etc.

\newcommand{\group}[2]{\ensuremath{(#1,#2)}}
\newcommand{\generates}[1]{\ensuremath{\langle {#1} \rangle}}
\newcommand{\generatest}[1]{\ensuremath{\left\langle {#1} \right\rangle}}
\newcommand{\perm}[1]{\ensuremath{\mathrm{Perm}(#1)}} % permutations
\newcommand{\aut}[1]{\ensuremath{\mathrm{Aut}(#1)}} % set of automorphisms (as a group)
\newcommand{\injhomeo}{\hookrightarrow}
\newcommand{\isomorph}{\approx}
\newcommand{\ideal}[1]{\ensuremath{\mathfrak{#1}}}

\newcommand{\collk}[1]{\ensuremath{{\mathcalfont{#1}}}} % collection
\newcommand{\classk}[1]{\ensuremath{\collk{#1}}} % class
\newcommand{\algk}[1]{\ensuremath{\mathalgfont{#1}}} % algebra
\newcommand{\metrics}[2]{\ensuremath{\langle {#1}, {#2}\rangle}} % metric space

\newcommand{\topol}[1]{\ensuremath{\mathfrak{#1}}} % topology
\newcommand{\topos}[2]{\ensuremath{{\langle {#1}, \topol{#2}\rangle}}} % topological space

\newcommand{\measu}[2]{\ensuremath{({#1}, {#2})}} % measurable space
\newcommand{\meas}[3]{\ensuremath{({#1}, {#2}, {#3})}} % measure space
\newcommand{\meast}[3]{\ensuremath{\left({#1}, {#2}, {#3}\right)}} % measure space

\newcommand{\powerset}{\mathcalfont{P}}
\newcommand{\field}{\mbox{\bf F}}
\newcommand\primefield[1]{\ensuremath{\field_{#1}}}
\newcommand\finitefield[2]{\ensuremath{\field_{{#1}^{#2}}}}
\newcommand\frobmap[2]{\ensuremath{\varphi_{{#1},{#2}}}}
\newcommand{\compl}[1]{\ensuremath{\tilde{#1}}} % set complement
\newcommand{\interior}[1]{\ensuremath{{#1}^\circ}} % set interior
\newcommand{\subsetset}[1]{\ensuremath{\mathcalfont{#1}}} % set of subsets

\newcommand{\pair}[2]{\ensuremath{{\langle {#1}, {#2}\rangle}}}
\newcommand{\innerp}[2]{\ensuremath{\langle{#1},{#2}\rangle}} % inner product
\newcommand{\innerpt}[2]{\ensuremath{\left\langle{#1},{#2}\right\rangle}} % inner product - tall version
\newcommand{\dimext}[2]{\ensuremath{[{#1}:{#2}]}}

\newcommand\restrict[2]{\ensuremath{{#1}|{#2}}}
\newcommand\algclosure[1]{\ensuremath{{#1}^\mathrm{a}}}
\newcommand\sepclosure[1]{\ensuremath{{#1}^\mathrm{sep}}}
\newcommand\maxabext[1]{\ensuremath{{#1}^\mathrm{ab}}}

\newcommand\ball[2]{\ensuremath{B(#1,#2)}}

% optimization

\newcommand\optfdk[2]{\ensuremath{{#1}^\mathrm{#2}}}
\newcommand\tildeoptfdk[2]{\ensuremath{{\tilde{#1}}^\mathrm{#2}}}
\newcommand\fobj{\optfdk{f}{obj}}
\newcommand\fie{\optfdk{f}{ie}}
\newcommand\feq{\optfdk{f}{eq}}
\newcommand\tildefobj{\tildeoptfdk{f}{obj}}
\newcommand\tildefie{\tildeoptfdk{f}{ie}}
\newcommand\tildefeq{\tildeoptfdk{f}{eq}}

\newcommand\xdomain{\ensuremath{\mathcalfont{X}}}
\newcommand\xobj{\optfdk{\xdomain}{obj}}
\newcommand\xie{\optfdk{\xdomain}{ie}}
\newcommand\xeq{\optfdk{\xdomain}{eq}}

\newcommand\optdomain{\ensuremath{\mathcalfont{D}}}
\newcommand\optfeasset{\ensuremath{\mathcalfont{F}}}

\def\DeltaSirDir{yes}
\newcommand\sdirletter[2]{\ifthenelse{\equal{\DeltaSirDir}{yes}}{\ensuremath{\Delta #1}}{\ensuremath{#2}}}
\newcommand\seqk[2]{\ensuremath{{#1}^{(#2)}}}
\newcommand\xseqk[1]{\seqk{x}{#1}}
\newcommand\nuseqk[1]{\seqk{\nu}{#1}}
\newcommand\lbdseqk[1]{\seqk{\lambda}{#1}}
\newcommand\sdir{\sdirletter{x}{v}}
\newcommand\sdirlbd{\sdirletter{\lambda}{\Delta \lambda}}
\newcommand\sdirnu{\sdirletter{\nu}{w}}
\newcommand\sdiry{\sdirletter{y}{\Delta y}}
\newcommand\ntsdir{\ensuremath{\sdir_\mathrm{nt}}}
\newcommand\pdsdir{\ensuremath{\sdir_\mathrm{pd}}}
\newcommand\ntsdirnu{\ensuremath{\sdirnu_\mathrm{nt}}}
\newcommand\pdsdirnu{\ensuremath{\sdirnu_\mathrm{pd}}}
\newcommand\pdsdirlbd{\ensuremath{\sdirlbd_\mathrm{pd}}}
\newcommand\pdsdiry{\ensuremath{\sdiry_\mathrm{pd}}}
\newcommand\sdirk[1]{\seqk{\sdir}{#1}}
\newcommand\slen{\ensuremath{t}}
\newcommand\slenk[1]{\seqk{\slen}{#1}}

\newcommand\onelineoptprob[3]{\ensuremath{%
	\mbox{#1}\;\;{#2}
%	\;\;
	\ifthenelse{\equal{#3}{}}{}{%
	\quad%
	\mbox{s.t.}\;%
	{#3}%
}}}

% mo, relation, sequence, indexed collection, etc.

\newcommand{\Mod}[1]{\ (\mathrm{mod}\ #1)}
\newcommand{\rel}{\mathbf{R}}
\newcommand{\relxy}[2]{{#1}\ \rel\ {#2}}
\newcommand{\seq}[1]{\ensuremath{{\left\langle{#1}\right\rangle}}}
\newcommand{\seqscr}[3]{\ensuremath{\seq{#1}_{#2}^{#3}}}
\newcommand{\indexedcol}[1]{\ensuremath{{\{{#1}\}}}}

% CLOSURE

%command for alg-closure that automatically adapts its 'bar' to the arg based on the args length (including '\')
\newcommand{\ols}[1]{\mskip.5\thinmuskip\overline{\mskip-.5\thinmuskip {#1} \mskip-.5\thinmuskip}\mskip.5\thinmuskip} % overline short
\newcommand{\olsi}[1]{\,\overline{\!{#1}}} % overline short italic
\makeatletter
\newcommand\closure[1]{\ensuremath{%
	\tctestifnum{\count@stringtoks{#1}>1} %checks if number of chars in arg > 1 (including '\')
	{\ols{#1}} %if arg is longer than just one char, e.g. \mathbb{Q}, \mathbb{F},...
	{\olsi{#1}} %if arg is just one char, e.g. K, L,...
}%
}
% FROM TOKCYCLE:
\long\def\count@stringtoks#1{\tc@earg\count@toks{\string#1}}
\long\def\count@toks#1{\the\numexpr-1\count@@toks#1.\tc@endcnt}
\long\def\count@@toks#1#2\tc@endcnt{+1\tc@ifempty{#2}{\relax}{\count@@toks#2\tc@endcnt}}
\def\tc@ifempty#1{\tc@testxifx{\expandafter\relax\detokenize{#1}\relax}}
\long\def\tc@earg#1#2{\expandafter#1\expandafter{#2}}
\long\def\tctestifnum#1{\tctestifcon{\ifnum#1\relax}}
\long\def\tctestifcon#1{#1\expandafter\tc@exfirst\else\expandafter\tc@exsecond\fi}
\long\def\tc@testxifx{\tc@earg\tctestifx}
\long\def\tctestifx#1{\tctestifcon{\ifx#1}}
\long\def\tc@exfirst#1#2{#1}
\long\def\tc@exsecond#1#2{#2}
\makeatother
%


% graphics - include figures

%\newcommand\includefig[2]{\yesnoexec{\pdfpsfrag}{\includegraphics[#2]{#1}}{\includegraphics[#2]{#1}}}
\newcommand\mypsfrag[2]{\yesnoexec{\pdfpsfrag}{\psfrag{#1}{#2}}{}}

\newcommand\puttwofigs[2]{%
	\begin{center}
	\hfill
	{#1}
	\hfill
	{#2}
	\hfill
	\hfill
	\ %
	\end{center}
}

\newcommand\puttworoundedfigs[3]{%
	\begin{center}
	\hfill
	{\roundedbox{#1}{#2}}
	\hfill
	{\roundedbox{#1}{#3}}
	\hfill
	\hfill
	\ %
	\end{center}
}

\newcommand\roundedbox[2]{%
\begin{tikzpicture}%
\sbox0{#2}%
\path[clip,draw,rounded corners=#1] (0,0) rectangle (\wd0,\ht0);%
\path (0.5\wd0,0.5\ht0) node[inner sep=0pt]{\usebox0};%
\end{tikzpicture}%
}

% COMMANDS ORIGINALLY INTENDED FOR FOILTEX

\newcommand\specialidxprefix{ZZ}
\newcommand\idximportant[1]{\index{\specialidxprefix-important!#1}}
\newcommand\idxrevisit[1]{\index{\specialidxprefix-revisit!#1}}
\newcommand\idxtodo[1]{\index{\specialidxprefix-todo!#1}}
\newcommand\idxfig[1]{\index{\specialidxprefix-figures!#1}}

% page label and reference
\newcommand\pagelabel{\phantomsection\label}

% enumeration - bullet points

\usepackage{enumitem}
\newcommand\bit{\begin{itemize}}
\newcommand\eit{\end{itemize}}
\newcommand\ibit{\begin{itemize}[leftmargin=1.5em]}
\newcommand\iitem{\item [-]}

% *my* theorems and such

\newcommand\theoremslikepostvspace{\vspace{-.5em}}
\newcommand\theoremslikepostvspacet{\vspace{-1em}}

\newcommand\notempty[1]{\ifthenelse{\not \equal{#1}{}}}

\newcommand\mybegin[3]{%
\notempty{#3}{%
\begin{#1}[#3]%
\index{#3}\index{#2!#3}%
\label{#1:#3}\pagelabel{page:#1:#3}%
}{%
\begin{#1}
}%
}

\newcommand\myend[1]{%
\end{#1}
\theoremslikepostvspace
}

\newenvironment{myaxiom}[1]{\mybegin{axiom}{axioms}{#1}}{\myend{axiom}}
\newenvironment{mylaw}[1]{\mybegin{law}{laws}{#1}}{\myend{law}}
\newenvironment{myprinciple}[1]{\mybegin{principle}{principles}{#1}}{\myend{principle}}
\newenvironment{mydefinition}[1]{\mybegin{definition}{definitions}{#1}}{\myend{definition}}
\newenvironment{mytheorem}[1]{\mybegin{theorem}{theorems}{#1}}{\myend{theorem}}
\newenvironment{mylemma}[1]{\mybegin{lemma}{lemmas}{#1}}{\myend{lemma}}
\newenvironment{myproposition}[1]{\mybegin{proposition}{propositions}{#1}}{\myend{proposition}}
\newenvironment{mycorollary}[1]{\mybegin{corollary}{corollaries}{#1}}{\myend{corollary}}
\newenvironment{myconjecture}[1]{\mybegin{conjecture}{conjectures}{#1}}{\myend{conjecture}}
\newenvironment{myinequality}[1]{\mybegin{inequality}{inequalities}{#1}}{\myend{inequality}}
\newenvironment{myformula}[1]{\mybegin{formula}{formula}{#1}}{\myend{formula}}
\newenvironment{myalgorithm}[1]{\mybegin{algorithm}{algorithms}{#1}}{\myend{algorithm}}

% proof env

\newtheorem{proofenv}{Proof}
\newenvironment{myproof}[1]{%
\item%
\begin{proofenv}%
\label{proof:#1}
\label{proof!#1}
(Proof for ``{#1}'' on page~\pageref{page:statement:#1})%
\end{proofenv}%
}{%
\vspace{.5em}
}

\newcommand\proofref[1]{%
%\textcolor{gray}%
{%
%(Refer to \hyperref[proof:#1]{Proof}~\ref{proof:#1}%
%(proved in \hyperref[proof:#1]{Proof}~\ref{proof:#1}%
(proof can be found in \hyperref[proof:#1]{Proof}~\ref{proof:#1}%
\pagelabel{page:statement:#1})%
}%
}

% terms for definitions, facts, special emphasis, (lemma, theorem, etc.) names

\newcommand\define[1]{\emph{\textcolor{blue}{#1}}}
\newcommand\fact[1]{\emph{\textcolor{Lime-Green}{#1}}}
\newcommand\cemph[1]{\emph{\textcolor{blue}{{#1}}}}
\newcommand\eemph[1]{\emph{\textcolor{crimson}{#1}}}
\newcommand\name[1]{\emph{\textcolor{chocolate}{#1}}}

% equations

\newenvironment{eqn}{%
%
\vspace{-.7em}
\[\\
%
}
{
\]\\
\vspace{-3.4em}\\
}

% when equation is too tall
\newenvironment{teqn}{%
%
\vspace{-.7em}
\[\\
%
}
{
\]\\
\vspace{-3.0em}\\
}

% when equation is too long
\newenvironment{leqn}{%
%
\vspace{-2.0em}
\[\\
%
}
{
\]\\
\vspace{-4.7em}\\
}

\newenvironment{eqna}{%
%
\vspace{-3.7em}
\begin{eqnarray*}\\
%
}
{
\end{eqnarray*}\\
\vspace{-4.5em}\\
}


%% FOR FOILTEX

%%%%%%%%%%%%%%%%%%%%%%%%%%%%%%%%%%%%%%%%%%%%%%%%%%%%%%%%%%%%%%%%%%%%%%%%%%%%%%%%
\yesnoexec{\foiltex}{
\usepackage{mathtools}


%% figure reference

\newcommand\figref[1]{the figure}
\newcommand\foilref[1]{on page~\pageref{foil:#1}}

%% enumeration - bullet points

\renewcommand\bit{\begin{itemize}[leftmargin=1.1em]}
\renewcommand\ibit{\begin{itemize}[leftmargin=2em]}

\newcommand\shrinkspacewithintheoremslikehalf{\vspace{-.5em}}
\newcommand\shrinkspacewithintheoremslike{\vspace{-1em}}
\newcommand\shrinkspacewithintheoremsliket{\vspace{-2em}}

\newcommand\vitem{\vfill \item}
\newcommand\vvitem{\vfill \vfill \item}
\newcommand\viitem{\vfill \iitem}

\newcommand\vfillt {\vfill\vfill}
\newcommand\vfillth{\vfill\vfill\vfill}
\newcommand\vfillf {\vfill\vfill\vfill\vfill}
\newcommand\vfillfi{\vfill\vfill\vfill\vfill\vfill}
\newcommand\vfills {\vfill\vfill\vfill\vfill\vfill\vfill}
\newcommand\vfillse{\vfill\vfill\vfill\vfill\vfill\vfill\vfill}
\newcommand\vfille {\vfill\vfill\vfill\vfill\vfill\vfill\vfill\vfill}
\newcommand\vfilln {\vfill\vfill\vfill\vfill\vfill\vfill\vfill\vfill\vfill}
\newcommand\vfillte{\vfill\vfill\vfill\vfill\vfill\vfill\vfill\vfill\vfill\vfill}


%% index
\usepackage{makeidx}
\makeindex

\setlength{\columnsep}{2em}
\newenvironment{theindex}{%
	\let\item\par
	\newcommand\idxitem{\par\hangindent 40pt}
	\newcommand\subitem{\vspace{-12pt} \idxitem \hspace*{20pt}}
	\newcommand\subsubitem{\vspace{-12pt} \idxitem \hspace*{30pt}}
	\newcommand\indexspace{\par \vskip 3pt \relax}
	\twocolumn
}{%
}

%% slide commands and formatting

\setlength{\parindent}{0in}
\newlength{\lift}
\setlength{\lift}{.5in}
\newlength{\pageheight}
%\setlength{\pageheight}{4in + \lift}
\setlength{\pageheight}{4.5in}

\newcommand{\raiselength}{0in}
\newcommand{\figraiselength}{0in}

\newcommand\labelfoilhead[1]{%
\myfoilhead{#1}%
\pagelabel{foil:#1}%
%\label{foil:#1}%
}

\newcommand\myfoilhead[1]{%
\foilhead{#1}%
\vspace{-\lift}%
}

\newcommand{\titlefoil}[2]{%
\myfoilhead{}%
%\index{#1}
\pagelabel{title-page:#2}
\thispagestyle{empty}%
%\addtocounter{page}{-1}%
%
\vfill%
\begin{center}%
\Large%
\bf%
{#1}
\end{center}%
\vfill%
\MyLogo{\LOGO\ - {\hyperref[title-page:#2]{#1}}}
}

\newcommand\LOGO{\talktitle}

\newcommand{\TITLEFOIL}[2]{%
\myfoilhead{}%
\pagelabel{super-title-page:#2}
%\index{#1}
\thispagestyle{empty}%
%\addtocounter{page}{-1}%
%
\vfill%
\begin{center}%
\huge%
\bf%
{#1}
\end{center}%
\vfill%
\renewcommand\LOGO{\talktitle\ - {\hyperref[super-title-page:#2]{#1}}}
\MyLogo{\LOGO}
}

\newcommand{\nntwocols}[6]{%
\begin{minipage}[t][\pageheight]{#1}\vspace*{0ex}{#6}\end{minipage}%
\hfill%
\begin{minipage}[t]{#2}%
\hfill
\raisebox{-\height+0.7\baselineskip - #4}%
{\hspace*{-3in}{#3}}%
#5
\end{minipage}%
}
\newcommand{\ntwocols}[7]{\nntwocols{#1}{#2}{\includegraphics[#4]{#3}}{#5}{#6}{#7}}

\newcommand{\nntwocolss}[6]{%
\begin{minipage}[t]{#2}%
\hfill
\raisebox{-\height+0.7\baselineskip - #4}%
{#5 \hspace*{-3in}{#3}\hfill\ %
}\end{minipage}%
\hfill%
\begin{minipage}[t][\pageheight]{#1}\vspace*{0ex}{#6}\end{minipage}%
}
\newcommand{\ntwocolss}[7]{\nntwocolss{#1}{#2}{\includegraphics[#4]{#3}}{#5}{#6}{#7}}

\newcommand{\ntwocolstwofigs}[8]{%
\begin{minipage}[t][\pageheight]{#1}\vspace*{0ex}{#8}\end{minipage}%
\hfill
\begin{minipage}[t]{#2}%
\begin{center}\raisebox{-\height+0.7\baselineskip - #7}{%
\ \hfill\includegraphics[#4]{#3}
}\end{center}\ \vfill\ \begin{center}\raisebox{0in}{%
\ \hfill\includegraphics[#6]{#5}
}\end{center}
\end{minipage}%
}

\newcommand{\ntwocolsstwofigs}[8]{%
\begin{minipage}[t]{#2}%
\begin{center}\raisebox{-\height+0.7\baselineskip - #7}{%
\includegraphics[#4]{#3}
\hfill\ %
}\end{center}%
\ \vfill\ %
\begin{center}\raisebox{0in}{%
\includegraphics[#6]{#5}
\hfill\ %
}\end{center}%
\end{minipage}%
%}
\hfill%
\begin{minipage}[t][\pageheight]{#1}%
\vspace*{0ex}
{
#8%
}%
\end{minipage}%
}

\newcommand\twocolsnormalsize[6]{\ntwocols{#1}{#2}{#3}{#4}{#5}{}{#6}}

\newcommand\twocols[6]{\twocolsnormalsize{#1}{#2}{#3}{#4}{#5}{\small #6}}

\newcommand{\twocolssnormalsize}[6]{\ntwocolss{#1}{#2}{#3}{#4}{#5}{}{#6}}
\newcommand{\twocolss}[6]{\twocolssnormalsize{#1}{#2}{#3}{#4}{#5}{\small #6}}


}{}
%%%%%%%%%%%%%%%%%%%%%%%%%%%%%%%%%%%%%%%%%%%%%%%%%%%%%%%%%%%%%%%%%%%%%%%%%%%%%%%%



%% PACKAGE IMPORTS

\yesnoexec{\pdfpsfrag}{%
	\usepackage{psfrag}%
	\yesnoexec{\reusepsfragpdf}{\usepackage[off]{auto-pst-pdf}}{\usepackage{auto-pst-pdf}}%
}{%
	\usepackage{graphicx}%
}%

